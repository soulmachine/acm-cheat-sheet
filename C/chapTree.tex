\chapter{树}

\section{二叉树的遍历} %%%%%%%%%%%%%%%%%%%%%%%%%%%%%%
\label{sec:binaryTreeTraversal}

在中序遍历中,一个节点的前驱,是其左子树的最右下角结点,后继,是其右子树的最左下角结点。

在后序遍历中,
\begindot
\item 若结点是根结点,则其后继为空;
\item 若结点是双亲的右子树,或是左子树但双亲无右子树,则其后继为双亲结点;
\item 若结点是双亲的左子树且双亲有右子树,则其后继为右子树按后序遍历的第一个结点
\myenddot


\begin{Codex}[label=binary_tree.cpp]
#include <iostream>
#include <stack>
#include <queue>

/** 结点的数据 */
typedef int tree_node_elem_t;
 /*
  *@struct
  *@brief 二叉树结点
  */
typedef struct binary_tree_node_t {
    binary_tree_node_t *left;   /* 左孩子*/
    binary_tree_node_t *right;   /* 右孩子*/
    tree_node_elem_t elem; /* 结点的数据*/
} binary_tree_node_t;

/**
  * @brief 先序遍历,递归.
  * @param[in] root 根结点
  * @param[in] visit 访问数据元素的函数指针
  * @return 无
  */
void pre_order_r(const binary_tree_node_t *root,
                 int (*visit)(const binary_tree_node_t*)) {
    if (root == NULL) return;

    visit(root);
    pre_order_r(root->left, visit);
    pre_order_r(root->right, visit);
}

/**
  * @brief 中序遍历,递归.
  */
void in_order_r(const binary_tree_node_t *root,
                int (*visit)(const binary_tree_node_t*)) {
    if(root == NULL) return;

    in_order_r(root->left, visit);
    visit(root);
    in_order_r(root->right, visit);
}

/**
  * @brief 后序遍历,递归.
  */
void post_order_r(const binary_tree_node_t *root,
                  int (*visit)(const binary_tree_node_t*)) {
    if(root == NULL) return;

    post_order_r(root->left, visit);
    post_order_r(root->right, visit);
    visit(root);
}

/**
 * @brief 先序遍历,非递归.
 */
void pre_order(const binary_tree_node_t *root,
               int (*visit)(const binary_tree_node_t*)) {
    const binary_tree_node_t *p;
    stack<const binary_tree_node_t *> s;

    p = root;

    if(p != NULL) s.push(p);

    while(!s.empty()) {
        p = s.top();
        s.pop();
        visit(p);

        if(p->right != NULL) s.push(p->right);
        if(p->left != NULL) s.push(p->left);
    }
}

/**
 * @brief 中序遍历,非递归.
 */
void in_order(const binary_tree_node_t *root,
              int (*visit)(const binary_tree_node_t*)) {
    const binary_tree_node_t *p;
    stack<const binary_tree_node_t *> s;

    p = root;

    while(!s.empty() || p!=NULL) {
        if(p != NULL) {
            s.push(p);
            p = p->left;
        } else {
            p = s.top();
            s.pop();
            visit(p);
            p = p->right;
        }
    }
}

/**
 * @brief 后序遍历,非递归.
 */
void post_order(const binary_tree_node_t *root,
                int (*visit)(const binary_tree_node_t*)) {
    /* p,正在访问的结点,q,刚刚访问过的结点*/
    const binary_tree_node_t *p, *q;
    stack<const binary_tree_node_t *> s;

    p = root;

    do {
        while(p != NULL) { /* 往左下走*/
            s.push(p);
            p = p->left;
        }
        q = NULL;
        while(!s.empty()) {
            p = s.top();
            s.pop();
            /* 右孩子不存在或已被访问,访问之*/
            if(p->right == q) {
                visit(p);
                q = p; /* 保存刚访问过的结点*/
            } else {
                /* 当前结点不能访问,需第二次进栈*/
                s.push(p);
                /* 先处理右子树*/
                p = p->right;
                break;
            }
        }
    } while(!s.empty());
}

/**
 * @brief 层次遍历,也即BFS.
 *
 * 跟先序遍历一模一样,唯一的不同是栈换成了队列
 */
void level_order(const binary_tree_node_t *root,
                int (*visit)(const binary_tree_node_t*)) {
    const binary_tree_node_t *p;
    queue<const binary_tree_node_t *> q;

    p = root;

    if(p != NULL) q.push(p);

    while(!q.empty()) {
        p = q.front();
        q.pop();
        visit(p);

        /*先左后右或先右后左无所谓*/
        if(p->left != NULL) q.push(p->left);
        if(p->right != NULL) q.push(p->right);
    }
}
\end{Codex}

\section{线索二叉树} %%%%%%%%%%%%%%%%%%%%%%%%%%%%%%
二叉树中存在很多空指针,可以利用这些空指针,指向其前驱或者后继。这种利用起来的空指针称为线索,这种改进后的二叉树称为线索二叉树(threaded binary tree)。

一棵n个结点的二叉树含有n+1个空指针。这是因为,假设叶子节点数为$n_0$,度为1的节点数为$n_1$,度为2的节点数为$n_2$,每个叶子节点有2个空指针,每个度为1的节点有1个空指针,则空指针的总数为$2n_0+n_1$,又有$n_0=n_2+1$(留给读者证明),因此空指针总数为$2n_0+n_1=n_0+n_2+1+n_1=n_0+n_1+n_2+1=n+1$。

在二叉树线索化过程中,通常规定,若无左子树,令lchild指向前驱,若无右子树,令rchild指向后继。还需要增加两个标志域表示当前指针是不是线索,例如ltag=1,表示lchild指向的是前驱,ltag=0,表示lchild指向的是左孩子,rtag类似。

二叉树的线索化,实质上就是遍历一棵树,只是在遍历的过程中,检查当前节点的左右指针是否为空,若为空,将它们改为指向前驱或后继的线索。

以中序线索二叉树为例,指针pre表示前驱,succ表示后继,如图~\ref{fig:threadedBinaryTree}所示。

\begin{center}
\includegraphics[width=300pt]{threaded-binary-tree.png} \\
\figcaption{中序线索二叉树}\label{fig:threadedBinaryTree}
\end{center}

在中序线索二叉树中,一个节点的前驱,是其左子树的最右下角结点,后继,是其右子树的最左下角结点。

中序线索二叉树的C语言实现如下。
\begin{Codex}[label=theaded_binary_tree.c]
/** @file threaded_binary_tree.c
  * @brief 线索二叉树.
  */
#include <stddef.h>    /* for NULL */
#include <stdio.h>

/* 结点数据的类型. */
typedef int elem_t;

 /**
  *@struct
  *@brief 线索二叉树结点.
  */
typedef struct tbt_node_t {
    int ltag; /** 1表示是线索,0表示是孩子 */
    int rtag; /** 1表示是线索,0表示是孩子 */
    struct tbt_node_t *left; /** 左孩子*/
    struct tbt_node_t *right; /** 右孩子*/
    elem_t elem; /** 结点所存放的数据*/
}tbt_node_t;

/* 内部函数 */
static void in_thread(tbt_node_t *p, tbt_node_t **pre);
static tbt_node_t *first(tbt_node_t *p);
static tbt_node_t *next(const tbt_node_t *p);

 /**
  * @brief 建立中序线索二叉树.
  * @param[in] root 树根
  * @return 无
  */
void create_in_thread(tbt_node_t *root) {
    /* 前驱结点指针*/
    tbt_node_t *pre=NULL;
    if(root != NULL) { /* 非空二叉树,线索化*/
        /* 中序遍历线索化二叉树*/
        in_thread(root, &pre);
        /* 处理中序最后一个结点*/
        pre->right = NULL;
        pre->rtag = 1;
    }
}


/**
  * @brief 在中序线索二叉树上执行中序遍历.
  * @param[in] root 树根
  * @param[in] visit 访问结点的数据的函数
  * @return 无
  */
void in_order(tbt_node_t *root, int(*visit)(tbt_node_t*)) {
    tbt_node_t *p;
    for(p = first(root); p != NULL; p = next(p)) {
        visit(p);
    }
}


 /*
  * @brief 中序线索化二叉树的主过程.
  * @param[in] p 当前要处理的结点
  * @param[inout] pre 当前结点的前驱结点
  * @return 无
  */
static void in_thread(tbt_node_t *p, tbt_node_t **pre) {
    if(p != NULL) {
        in_thread(p->left, pre); /* 线索化左子树 */
        if(p->left == NULL) {  /* 左子树为空,建立前驱 */
            p->left = *pre;
            p->ltag = 1;
        }
        /* 建立前驱结点的后继线索 */
        if((*pre) != NULL &&
            (*pre)->right == NULL) {
            (*pre)->right = p;
            (*pre)->rtag = 1;
        }
        *pre = p; /* 更新前驱 */
        in_thread(p->right, pre); /* 线索化右子树 */
    }
}

 /*
  * @brief 寻找线索二叉树的中序下的第一个结点.
  * @param[in] p 线索二叉树中的任意一个结点
  * @return 此线索二叉树的第一个结点
  */
static tbt_node_t *first(tbt_node_t *p) {
    if(p == NULL)  return NULL;

    while(p->ltag == 0) {
        p = p->left;  /* 最左下结点,不一定是叶结点*/
    }
    return p;
}

 /*
  * @brief 求中序线索二叉树中某结点的后继.
  * @param[in] p 某结点
  * @return p的后继
  */
static tbt_node_t *next(const tbt_node_t *p) {
    if(p->rtag == 0) {
        return first(p->right);
    } else {
        return p->right;
    }
}
\end{Codex}

中序线索二叉树最简单,在中序线索的基础上稍加修改就可以实现先序,后续就要再费点心思了。


\section{Morris Traversal} %%%%%%%%%%%%%%%%%%%%%%%%%%%%%%
通过前面第\S \ref{sec:binaryTreeTraversal}节,我们知道,实现二叉树的前序(preorder)、中序(inorder)、后序(postorder)遍历有两个常用的方法,一是递归(recursive),二是栈(stack+iterative)。这两种方法都是O(n)的空间复杂度。

而Morris Traversal只需要O(1)的空间复杂度。这种算法跟线索二叉树很像,不过Morris Traversal一边建线索,一边访问数据,访问完后销毁线索,保持二叉树不变。

\subsection{Morris中序遍历}
Morris中序遍历的步骤如下:
\begin{enumerate}
\item 初始化当前节点cur为root节点
\item 如果cur没有左孩子,则输出当前节点并将其右孩子作为当前节点,即cur = cur->right。
\item 如果cur有左孩子,则寻找cur的前驱,即cur的左子树的最右下角结点。\\
   a) 如果前驱节点的右孩子为空,将它的右孩子指向当前节点,当前节点更新为当前节点的左孩子。\\
   b) 如果前驱节点的右孩子为当前节点,将它的右孩子重新设为空(恢复树的形状),输出当前节点,当前节点更新为当前节点的右孩子。
\item 重复2、3步骤,直到cur为空。
\end{enumerate}
如图~\ref{fig:inorderMorris}所示,cur表示当前节点,深色节点表示该节点已输出。

\begin{center}
\includegraphics[width=360pt]{inorder-morris-traversal.png} \\
\figcaption{Morris中序遍历}\label{fig:inorderMorris}
\end{center}

C语言实现见第\S\ref{sec:morrisTraversalImpl}节。

\subsubsection{相关的题目}
\begindot
\item Leet Code - Binary Tree Inorder Traversal, \myurl{http://leetcode.com/onlinejudge\#question_94}
\myenddot


\subsection{Morris先序遍历}
Morris先序遍历的步骤如下:
\begin{enumerate}
\item 初始化当前节点cur为root节点
\item 如果cur没有左孩子,则输出当前节点并将其右孩子作为当前节点,即cur = cur->right。
\item 如果cur有左孩子,则寻找cur的前驱,即cur的左子树的最右下角结点。\\
   a) 如果前驱节点的右孩子为空,将它的右孩子指向当前节点,\textbf{输出当前节点(在这里输出,这是与中序遍历唯一的不同点)}当前节点更新为当前节点的左孩子。\\
   b) 如果前驱节点的右孩子为当前节点,将它的右孩子重新设为空(恢复树的形状),\sout{输出当前节点,}当前节点更新为当前节点的右孩子。
\item 重复2、3步骤,直到cur为空。
\end{enumerate}
如图~\ref{fig:preorderMorris}所示。

\begin{center}
\includegraphics[width=360pt]{preorder-morris-traversal.png} \\
\figcaption{Morris先序遍历}\label{fig:preorderMorris}
\end{center}

C语言实现见第\S\ref{sec:morrisTraversalImpl}节。


\subsection{Morris后序遍历}
Morris后续遍历稍微复杂,需要建立一个临时节点dump,令其左孩子是root,并且还需要一个子过程,就是倒序输出某两个节点之间路径上的所有节点。

Morris后序遍历的步骤如下:
\begin{enumerate}
\item 初始化当前节点cur为root节点
\item 如果cur没有左孩子,则\sout{输出当前节点并}将其右孩子作为当前节点,即cur = cur->right。
\item 如果cur有左孩子,则寻找cur的前驱,即cur的左子树的最右下角结点。\\
   a) 如果前驱节点的右孩子为空,将它的右孩子指向当前节点,当前节点更新为当前节点的左孩子。\\
   b) 如果前驱节点的右孩子为当前节点,将它的右孩子重新设为空(恢复树的形状),\sout{输出当前节点,}\textbf{倒序输出从当前节点的左孩子到该前驱节点这条路径上的所有节点。}当前节点更新为当前节点的右孩子。
\item 重复2、3步骤,直到cur为空。
\end{enumerate}
如图~\ref{fig:postorderMorris}所示。

\begin{center}
\includegraphics[width=360pt]{postorder-morris-traversal.png} \\
\figcaption{Morris后序遍历}\label{fig:postorderMorris}
\end{center}

C语言实现见第\S\ref{sec:morrisTraversalImpl}节。


\subsection{C语言实现}
\label{sec:morrisTraversalImpl}
\begin{Codex}[label=morris_traversal.c]
/** @file morris_traversal.c
 * @brief Morris遍历算法.
 */
#include<stdio.h>
#include<stdlib.h>

/* 结点数据的类型. */
typedef int elem_t;

/**
 *@struct
 *@brief 二叉树结点.
 */
typedef struct bt_node_t {
    elem_t elem; /* 节点的数据 */
    struct bt_node_t *left; /* 左孩子 */
    struct bt_node_t *right; /* 右孩子 */
} bt_node_t;

/**
 * @brief 中序遍历,Morris算法.
 * @param[in] root 根节点
 * @param[in] visit 访问函数
 * @return 无
 */
void in_order_morris(bt_node_t *root, int(*visit)(bt_node_t*)) {
    bt_node_t *cur, *prev;

    cur = root;
    while (cur != NULL ) {
        if (cur->left == NULL ) {
            visit(cur);
            prev = cur;
            cur = cur->right;
        } else {
            /* 查找前驱 */
            bt_node_t *node = cur->left;
            while (node->right != NULL && node->right != cur)
                node = node->right;

            if (node->right == NULL ) { /* 还没线索化,则建立线索 */
                node->right = cur;
                /* prev = cur; 不能有这句,cur还没有被访问 */
                cur = cur->left;
            } else {    /* 已经线索化,则访问节点,并删除线索  */
                visit(cur);
                node->right = NULL;
                prev = cur;
                cur = cur->right;
            }
        }
    }
}

/**
 * @brief 先序遍历,Morris算法.
 * @param[in] root 根节点
 * @param[in] visit 访问函数
 * @return 无
 */
void pre_order_morris(bt_node_t *root, int (*visit)(bt_node_t*)) {
    bt_node_t *cur, *prev;

    cur = root;
    while (cur != NULL ) {
        if (cur->left == NULL ) {
            visit(cur);
            prev = cur; /* cur刚刚被访问过 */
            cur = cur->right;
        } else {
            /* 查找前驱 */
            bt_node_t *node = cur->left;
            while (node->right != NULL && node->right != cur)
                node = node->right;

            if (node->right == NULL ) { /* 还没线索化,则建立线索 */
                visit(cur); /* 仅这一行的位置与中序不同 */
                node->right = cur;
                prev = cur; /* cur刚刚被访问过 */
                cur = cur->left;
            } else {    /* 已经线索化,则删除线索  */
                node->right = NULL;
                /* prev = cur; 不能有这句,cur已经被访问 */
                cur = cur->right;
            }
        }
    }
}


static void reverse(bt_node_t *from, bt_node_t *to);
static void visit_reverse(bt_node_t* from, bt_node_t *to,
        int (*visit)(bt_node_t*));
/**
 * @brief 后序遍历,Morris算法.
 * @param[in] root 根节点
 * @param[in] visit 访问函数
 * @return 无
 */
void post_order_morris(bt_node_t *root, int (*visit)(bt_node_t*)) {
    bt_node_t dummy = { 0, NULL, NULL };
    bt_node_t *cur, *prev = NULL;

    dummy.left = root;
    cur = &dummy;
    while (cur != NULL ) {
        if (cur->left == NULL ) {
            prev = cur; /* 必须要有 */
            cur = cur->right;
        } else {
            bt_node_t *node = cur->left;
            while (node->right != NULL && node->right != cur)
                node = node->right;

            if (node->right == NULL ) { /* 还没线索化,则建立线索 */
                node->right = cur;
                prev = cur; /* 必须要有 */
                cur = cur->left;
            } else { /* 已经线索化,则访问节点,并删除线索  */
                visit_reverse(cur->left, prev, visit);  // call print
                prev->right = NULL;
                prev = cur; /* 必须要有 */
                cur = cur->right;
            }
        }
    }
}

/*
 * @brief 逆转路径.
 * @param[in] from from
 * @param[to] to to
 * @return 无
 */
static void reverse(bt_node_t *from, bt_node_t *to) {
    bt_node_t *x = from, *y = from->right, *z;
    if (from == to) return;

    while (x != to) {
        z = y->right;
        y->right = x;
        x = y;
        y = z;
    }
}

/*
 * @brief  访问逆转后的路径上的所有结点.
 * @param[in] from from
 * @param[to] to to
 * @return 无
 */
static void visit_reverse(bt_node_t* from, bt_node_t *to,
        int (*visit)(bt_node_t*)) {
    bt_node_t *p = to;
    reverse(from, to);

    while (1) {
        visit(p);
        if (p == from)
            break;
        p = p->right;
    }

    reverse(to, from);
}

/*
 * @brief 分配一个新节点.
 * @param[in] e 新节点的数据
 * @return 新节点
 */
bt_node_t* new_node(int e) {
    bt_node_t* node = (bt_node_t*) malloc(sizeof(bt_node_t));
    node->elem = e;
    node->left = NULL;
    node->right = NULL;

    return (node);
}

static int print(bt_node_t *node) {
    printf(" %d ", node->elem);
    return 0;
}

/* test */
int main() {
    /* 构造的二叉树如下
       1
     /   \
    2      3
  /  \
4     5
     */
    bt_node_t *root = new_node(1);
    root->left = new_node(2);
    root->right = new_node(3);
    root->left->left = new_node(4);
    root->left->right = new_node(5);

    in_order_morris(root, print);
    printf("\n");
    pre_order_morris(root, print);
    printf("\n");
    post_order_morris(root, print);
    printf("\n");

    return 0;
}
\end{Codex}


\section{重建二叉树} %%%%%%%%%%%%%%%%%%%%%%%%%%%%%%
\begin{Codex}[label=binary_tree_rebuild.c]
#include <stdio.h>
#include <stdlib.h>
#include <string.h>
#include <stddef.h>
/**
 * @brief 给定前序遍历和中序遍历,输出后序遍历.
 *
 * @param[in] pre 前序遍历的序列
 * @param[in] in 中序遍历的序列
 * @param[in] n 序列的长度
 * @param[out] post 后续遍历的序列
 * @return 无
 */
void build_post(const char * pre, const char *in, const int n, char *post) {
    int left_len = strchr(in, pre[0]) - in;
    if(n <= 0) return;
    
    build_post(pre + 1, in, left_len, post);
    build_post(pre + left_len + 1, in + left_len + 1,
            n - left_len - 1, post + left_len);
    post[n - 1] = pre[0];
}

#define MAX  64
// 测试
// BCAD CBAD,输出 CDAB
// DBACEGF ABCDEFG,输出 ACBFGED
void build_post_test() {
    char pre[MAX] = {0};
    char in[MAX] = {0};
    char post[MAX] = {0};
    int n;

    scanf("%s%s", pre, in);
    n = strlen(pre);

    build_post(pre, in, n, post);
    printf("%s\n", post);
}

/* 结点数据的类型. */
typedef char elem_t;

/**
 *@struct
 *@brief 二叉树结点.
 */
typedef struct bt_node_t {
    elem_t elem; /* 节点的数据 */
    struct bt_node_t *left; /* 左孩子 */
    struct bt_node_t *right; /* 右孩子 */
} bt_node_t;

/**
 * @brief 给定前序遍历和中序遍历,重建二叉树.
 *
 * @param[in] pre 前序遍历的序列
 * @param[in] in 中序遍历的序列
 * @param[in] n 序列的长度
 * @param[out] root 根节点
 * @return 无
 */
void rebuild(const char *pre, const char *in, int n, bt_node_t **root) {
    int left_len;
    // 检查终止条件
    if (n <= 0 || pre == NULL || in == NULL)
        return;
    //获得前序遍历的第一个结点
    *root = (bt_node_t*) malloc(sizeof(bt_node_t));
    (*root)->elem = *pre;
    (*root)->left = NULL;
    (*root)->right = NULL;

    left_len = strchr(in, pre[0]) - in;
    //重建左子树
    rebuild(pre + 1, in, left_len, &((*root)->left));
    //重建右子树
    rebuild(pre + left_len + 1, in + left_len + 1, n - left_len - 1,
            &((*root)->right));
}

void print_post_order(const bt_node_t *root) {
    if(root != NULL) {
        print_post_order(root->left);
        print_post_order(root->right);
        printf("%c", root->elem);
    }
}

void rebuild_test() {
    char pre[MAX] = { 0 };
    char in[MAX] = { 0 };
    int n;
    bt_node_t *root;
    scanf("%s%s", pre, in);
    n = strlen(pre);
    
    rebuild(pre, in, n, &root);
    print_post_order(root);
}

int main() {
    build_post_test();
    rebuild_test();
    return 0;
}
\end{Codex}


\section{堆} %%%%%%%%%%%%%%%%%%%%%%%%%%%%%%

\subsection{原理和实现}
C++可以直接使用\fn{priority_queue}。

\begin{Codex}[label=heap.c]
/** @file heap.c
 * @brief 堆,默认为小根堆,即堆顶为最小.
 * @author soulmachine@gmail.com
 */
#include <stdlib.h>  /* for malloc() */
#include <string.h>  /* for memcpy() */

typedef int heap_elem_t; // 元素的类型

/**
 * @struct
 * @brief 堆的结构体
 */
typedef struct heap_t {
    int     size;   /** 实际元素个数 */
    int     capacity; /** 容量,以元素为单位 */
    heap_elem_t  *elems;   /** 堆的数组 */
    int (*cmp)(const heap_elem_t*, const heap_elem_t*);   /** 元素的比较函数 */
}heap_t;


/** 基本类型(如int, long, float, double)的比较函数 */
int cmp_int(const int *x, const int *y) {
    const int sub = *x - *y;
    if(sub > 0) {
        return 1;
    } else if(sub < 0) {
        return -1;
    } else {
        return 0;
    }
}

/**
 * @brief 创建堆.
 * @param[out] capacity 初始容量
 * @param[in] cmp cmp 比较函数,小于返回-1,等于返回0
 *            大于返回1,反过来则是大根堆
 * @return 成功返回堆对象的指针,失败返回 NULL
 */
heap_t* heap_create(const int capacity,
        int (*cmp)(const heap_elem_t*, const heap_elem_t*)) {
    heap_t *h = (heap_t*)malloc(sizeof(heap_t));
    h->size = 0;
    h->capacity = capacity;
    h->elems = (heap_elem_t*)malloc(capacity * sizeof(heap_elem_t));
    h->cmp = cmp;

    return h;
}

/**
 * @brief 销毁堆.
 * @param[inout] h 堆对象的指针
 * @return 无
 */
void heap_destroy(heap_t *h) {
    free(h->elems);
    free(h);
}


/**
 * @brief 判断堆是否为空.
 * @param[in] h 堆对象的指针
 * @return 是空,返回 1,否则返回 0
 */
int heap_empty(const heap_t *h) {
    return h->size == 0;
}

/**
 * @brief 获取元素个数.
 * @param[in] s 堆对象的指针
 * @return 元素个数
 */
int heap_size(const heap_t *h) {
    return h->size;
}

/*
 * @brief 小根堆的自上向下筛选算法.
 * @param[in] h 堆对象的指针
 * @param[in] start 开始结点
 * @return 无
 */
void heap_sift_down(const heap_t *h, const int start) {
    int i = start;
    int j;
    const heap_elem_t tmp = h->elems[start];

    for(j = 2 * i + 1; j < h->size; j = 2 * j + 1) {
        if(j < (h->size - 1) &&
            // h->elems[j] > h->elems[j + 1]
            h->cmp(&(h->elems[j]), &(h->elems[j + 1])) > 0) {
                j++; /* j 指向两子女中小者*/
        }
        // tmp <= h->data[j]
        if(h->cmp(&tmp, &(h->elems[j])) <= 0) {
            break;
        } else {
            h->elems[i] = h->elems[j];
            i = j;
        }
    }
    h->elems[i] = tmp;
}

/*
 * @brief 小根堆的自下向上筛选算法.
 * @param[in] h 堆对象的指针
 * @param[in] start 开始结点
 * @return 无
 */
void heap_sift_up(const heap_t *h, const int start) {
    int j = start;
    int i= (j - 1) / 2;
    const heap_elem_t tmp = h->elems[start];

    while(j > 0) {
        // h->data[i] <= tmp
        if(h->cmp(&(h->elems[i]), &tmp) <= 0) {
            break;
        } else {
            h->elems[j] = h->elems[i];
            j = i;
            i = (i - 1) / 2;
        }
    }
    h->elems[j] = tmp;
}

/**
 * @brief 添加一个元素.
 * @param[in] h 堆对象的指针
 * @param[in] x 要添加的元素
 * @return 无
 */
void heap_push(heap_t *h, const heap_elem_t x) {
    if(h->size == h->capacity) { /*已满,重新分配内存*/
        heap_elem_t* tmp =
            (heap_elem_t*)realloc(h->elems, h->capacity * 2 * sizeof(heap_elem_t));
        h->elems = tmp;
        h->capacity *= 2;
    }

    h->elems[h->size] = x;
    h->size++;

    heap_sift_up(h, h->size - 1);
}

/**
 * @brief 弹出堆顶元素.
 * @param[in] h 堆对象的指针
 * @return 无
 */
void heap_pop(heap_t *h) {
    h->elems[0] = h->elems[h->size - 1];
    h->size --;
    heap_sift_down(h, 0);
}

/**
 * @brief 获取堆顶元素.
 * @param[in] h 堆对象的指针
 * @return 堆顶元素
 */
heap_elem_t heap_top(const heap_t *h) {
    return h->elems[0];
}
\end{Codex}


\subsection{最小的N个和} %%%%%%%%%%%%%%%%%%%%%%%%%%%%%%
\subsubsection{描述}
有两个长度为$N$的序列 A 和 B,在 A 和 B 中各任取一个数可以得到 $N^2$ 个和,求这$N^2$ 个和中最小的$N$个。

\subsubsection{输入}
第一行输入一个正整数$N$;第二行N个整数$A_i$ 且$A_i \leq 10^9$;第三行$N$个整数$B_i$,且$Bi \leq 10^9$。

\subsubsection{输出}
输出仅一行,包含$N$个整数,从小到大输出这$N$个最小的和,相邻数字之间用空格隔开。

\subsubsection{样例输入}
\begin{Code}
5
1 3 2 4 5 
6 3 4 1 7
\end{Code}

\subsubsection{样例输出}
\begin{Code}
2 3 4 4 5
\end{Code}

\subsubsection{分析}
由于数据太大,有$N^2$个和,不能通过先求和再排序的方式来求解,这个时候就要用到堆了。

首先将A,B两数组排序,我们可以建立这样一个有序表:
\begin{eqnarray}
A_1+B_1<A_1+B_2<A_1+B_3< &...& <A_1+B_N \nonumber \\
A_2+B_1<A_2+B_2<A_2+B_3< &...& <A_2+B_N \nonumber \\
& ...  \nonumber \\
A_N+B_1<A_N+B_2<A_N+B_3< &...& <A_N+B_N \nonumber
\end{eqnarray}

首先将\fn{A[i] + B[0]}压入堆中,设每次出堆的元素为\fn{sum=A[a]+B[b]},则将\fn{A[a]+B[b+1]}入堆,这样可以保证前$N$个出堆的元素为最小的前$N$项。在实现的时候,可以不用保存B数组的下标,通过\fn{sum-B[b]+B[b+1]}来替换\fn{A[a]+B[b+1]}来节省空间。

\subsubsection{代码}
\begin{Codex}[label=sequence.c]
/* wikioi 1245 最小的N个和,http://www.wikioi.com/problem/1245/  */
#include <stdio.h>
#include <stdlib.h>  /* for malloc() */
#include <string.h>  /* for memcpy() */

#define MAXN 100000

int N;
int a[MAXN], b[MAXN];

typedef struct node_t {
    int sum;
    int b; /* sum=a[i]+b[b] */
} node_t;

typedef node_t heap_elem_t; // 元素的类型
/* 等价于复制粘贴,这里为了节约篇幅,使用include,在OJ上提交时请用复制粘贴 */
#include "heap.c"

void k_merge() {
    heap_t *h;
    int i;
    node_t tmp;

    qsort(a, N, sizeof(int), cmp_int);
    qsort(b, N, sizeof(int), cmp_int);
    h = heap_create(N, cmp_node);

    for (i = 0; i < N; i++) {
        tmp.sum = a[i]+b[0];
        tmp.b = 0;
        heap_push(h, tmp);
    }

    for (i = 0; i < N; i++) {
        tmp = heap_top(h); heap_pop(h);
        printf("%d ", tmp.sum);
        tmp.sum = tmp.sum - b[tmp.b] + b[tmp.b + 1];
        tmp.b++;
        heap_push(h, tmp);
    }

    heap_destroy(h);
    return;
}

int main() {
    int i;

    scanf("%d", &N);
    for (i = 0; i < N; i++) {
        scanf("%d", &a[i]);
    }
    for (i = 0; i < N; i++) {
        scanf("%d", &b[i]);
    }

    k_merge();
    return 0;
}
\end{Codex}

\subsubsection{相关的题目}
与本题相同的题目:
\begindot
\item wikioi 1245 最小的N个和, \myurl{http://www.wikioi.com/problem/1245/}
\myenddot

与本题相似的题目:
\begindot
\item  POJ 2442 Sequence, \myurl{http://poj.org/problem?id=2442}
\myenddot


\section{并查集} %%%%%%%%%%%%%%%%%%%%%%%%%%%%%%

\subsection{原理和实现}
通常用树双亲表示作为并查集的存储结构。每个集合以一棵树表示,数组元素的下标代表元素名,根结点的双亲指针为一个负数,表示集合的元素的个数。如图~\ref{fig:ufs1}、图~\ref{fig:ufs2}和图~\ref{fig:ufs3}所示。

\begin{center}
\includegraphics[width=280pt]{ufs1.png}\\
\figcaption{并查集的初始化}\label{fig:ufs1}
\end{center}

\begin{center}
\includegraphics[width=280pt]{ufs2.png}\\
\figcaption{用树表示并查集}\label{fig:ufs2}
\end{center}

\begin{center}
\includegraphics[width=380pt]{ufs3.png}\\
\figcaption{两个集合的并}\label{fig:ufs3}
\end{center}

并查集的C语言实现如下。

\begin{Codex}[label=ufs.c]
#include <stdlib.h>

/** 并查集. */
typedef struct ufs_t {
    int *p;     /** 树的双亲表示法 */
    int size;   /** 大小. */
} ufs_t;

/**
 * @brief 创建并查集.
 * @param[in] n 数组的容量
 * @return 并查集
 */
ufs_t* ufs_create(int n) {
    ufs_t *ufs = (ufs_t*)malloc(sizeof(ufs_t));
    int i;
    ufs->p = (int*)malloc(n * sizeof(int));
    for(i = 0; i < n; i++)
        ufs->p[i] = -1;
    return ufs;
}

/**
 * @brief 销毁并查集.
 * @param[in] ufs 并查集
 * @return 无
 */
void ufs_destroy(ufs_t *ufs) {
    free(ufs->p);
    free(ufs);
}

/**
 * @brief Find操作,带路径压缩,递归版.
 * @param[in] s 并查集
 * @param[in] x 要查找的元素
 * @return 包含元素x的树的根
 */
int ufs_find(ufs_t *ufs, int x) {
    if (ufs->p[x] < 0) return x; // 终止条件

    return ufs->p[x] = ufs_find(ufs, ufs->p[x]); /* 回溯时的压缩路径 */
}

/** Find操作,朴素版, deprecated. */
static int ufs_find_naive(ufs_t *ufs, int x) {
    while (ufs->p[x] >= 0) {
        x = ufs->p[x];
    }
    return x;
}

/** Find操作,带路径压缩,迭代版. */
static int ufs_find_iterative(ufs_t *ufs, int x) {
    int oldx = x; /* 记录原始x */
    while (ufs->p[x] >= 0) {
        x = ufs->p[x];
    }
    while (oldx != x) {
        int temp = ufs->p[oldx];
        ufs->p[oldx] = x;
        oldx = temp;
    }
    return x;
}

/**
 * @brief Union操作,将y并入到x所在的集合.
 * @param[in] s 并查集
 * @param[in] x 一个元素
 * @param[in] y 另一个元素
 * @return 如果二者已经在同一集合,并失败,返回-1,否则返回0
 */
int ufs_union(ufs_t *ufs, int x, int y) {
    const int rx = ufs_find(ufs, x);
    const int ry = ufs_find(ufs, y);
    if(rx == ry) return -1;

    ufs->p[rx] += ufs->p[ry];
    ufs->p[ry] = rx;
    return 0;
}

/**
 * @brief 获取元素所在的集合的大小
 * @param[in] ufs 并查集
 * @param[in] x 元素
 * @return 元素所在的集合的大小
 */
int ufs_set_size(ufs_t *ufs, int x) {
    const int rx = ufs_find(ufs, x);
    return -ufs->p[rx];
}
\end{Codex}


\subsection{病毒感染者} %%%%%%%%%%%%%%%%%%%%%%%%%%%%%%
\subsubsection{描述}
一个学校有$n$个社团,一个学生能同时加入不同的社团。由于社团内的同学们交往频繁,如果一个学生感染了病毒,该社团的所有学生都会感染病毒。现在0号学生感染了病毒,问一共有多少个人感染了病毒。

\subsubsection{输入}
输入包含多组测试用例。每个测试用例,第一行包含两个整数$n$,$m$,$n$表示学生个数,$m$表示社团个数。假设$0 < n \leq 30000, 0 \leq m \leq 500$。每个学生从0到$n-1$编号。接下来是$m$行,每行开头是一个整数k,表示该社团的学生个数,接着是$k$个整数表示该社团的学生编号。最后一个测试用例,$n=0,m=0$,表示输入结束。

\subsubsection{输出}
对每个测试用例,输出感染了病毒的学生数目。

\subsubsection{样例输入}
\begin{Code}
100 4
2 1 2
5 10 13 11 12 14
2 0 1
2 99 2
200 2
1 5
5 1 2 3 4 5
1 0
0 0
\end{Code}

\subsubsection{样例输出}
\begin{Code}
4
1
1
\end{Code}

\subsubsection{分析}
非常简单的并查集题目。

\subsubsection{代码}
\begin{Codex}[label=suspects.c]
/* POJ 1611 The Suspects, http://poj.org/problem?id=1611 */
#include <stdio.h>

#define MAXN 30000

/* 等价于复制粘贴,这里为了节约篇幅,使用include,在OJ上提交时请用复制粘贴 */
#include "ufs.c"  /* 见“树->并查集”这节 */

int main() {
    int n, m, k;
    while (scanf("%d%d", &n, &m) && n > 0) {
        ufs_t *ufs = ufs_create(MAXN);
        while (m--) {
            int x, y; /* 两个学生 */
            int rx, ry; /* x, y 所属的集合的根 */
            scanf("%d", &k);

            k--;
            scanf("%d", &x);
            rx = ufs_find(ufs, x);
            while (k--) {
                scanf("%d", &y);
                ry = ufs_find(ufs, y);
                ufs_union(ufs, rx, ry);  /* 只要是跟x同一个集合的都并进去 */
            }
        }
        /* 最后搜索0属于哪个集合,这个集合有多少人 */
        printf("%d\n", ufs_set_size(ufs, 0));
        ufs_destroy(ufs);
    }
    return 0;
}
\end{Codex}

\subsubsection{相关的题目}
与本题相同的题目:
\begindot
\item POJ 1611 The Suspects, \myurl{http://poj.org/problem?id=1611}
\myenddot

与本题相似的题目:
\begindot
\item  None
\myenddot


\subsection{两个黑帮} %%%%%%%%%%%%%%%%%%%%%%%%%%%%%%
\subsubsection{描述}
Tadu城市有两个黑帮帮派,已知有$N$黑帮分子,从1到$N$编号,每个人至少属于一个帮派。每个帮派至少有一个人。给你$M$条信息,有两类信息:
\begindot
\item D a b,明确告诉你,a和b属于不同的帮派 
\item A a b,问你,a和b是否属于不同的帮派
\myenddot

\subsubsection{输入}
第一行是一个整数$T$,表示有$T$组测试用例。每组测试用例的第一行是两个整数$N$和$M$,接下来是$M$行,每行包含一条消息。

\subsubsection{输出}
对每条消息"A a b",基于当前获得的信息,输出判断。答案是"In the same gang.", "In different gangs." 和 "Not sure yet."中的一个。

\subsubsection{样例输入}
\begin{Code}
1
5 5
A 1 2
D 1 2
A 1 2
D 2 4
A 1 4
\end{Code}

\subsubsection{样例输出}
\begin{Code}
Not sure yet.
In different gangs.
In the same gang.
\end{Code}

\subsubsection{分析}
把不在一个集合的节点直接用并查集合并在一起。这样的话,如果询问的2个节点在同一个并查集里面,那么它们之间的关系是确定的,否则无法确定它们的关系。

现在还有一个问题是,在同一个集合里面的2个节点是敌对关系还是朋友关系?可以给每个节点另外附加个信息,记录其距离集合根节点的距离。如果,询问的2个节点距离其根节点的距离都是奇数或者都是偶数,那么这2个节点是朋友关系,否则是敌对关系。

\subsubsection{代码}
\begin{Codex}[label=two_gangs.c]
/* POJ 1703 Find them, Catch them, http://poj.org/problem?id=1703 */
#include <stdio.h>
#include <stdlib.h>

#define MAXN 1000001

/** 并查集. */
typedef struct ufs_t {
    int *p;     /** 树的双亲表示法 */
    int *dist;  /** 到根节点的距离的奇偶性 */
    int size;   /** 大小. */
} ufs_t;

/**
 * @brief 创建并查集.
 * @param[in] ufs 并查集
 * @param[in] ufs 并查集
 * @param[in] n 数组的容量
 * @return 并查集
 */
ufs_t* ufs_create(int n) {
    int i;
    ufs_t *ufs = (ufs_t*)malloc(sizeof(ufs_t));
    ufs->p = (int*)malloc(n * sizeof(int));
    ufs->dist = (int*)malloc(n * sizeof(int));
    for(i = 0; i < n; i++) {
        ufs->p[i] = -1;
        ufs->dist[i] = 0;
    }
    return ufs;
}

/**
 * @brief 销毁并查集.
 * @param[in] ufs 并查集
 * @return 无
 */
void ufs_destroy(ufs_t *ufs) {
    free(ufs->p);
    free(ufs->dist);
    free(ufs);
}

/**
 * @brief Find操作,带路径压缩,递归版.
 * @param[in] s 并查集
 * @param[in] x 要查找的元素
 * @return 包含元素x的树的根
 */
int ufs_find(ufs_t *ufs, int x) {
    if (ufs->p[x] < 0) return x; // 终止条件

    const int parent = ufs->p[x];
    ufs->p[x] = ufs_find(ufs, ufs->p[x]); /* 回溯时的压缩路径 */
    ufs->dist[x] = (ufs->dist[x] + ufs->dist[parent]) % 2;
    return ufs->p[x];
}

/**
 * @brief Union操作,将root2并入到root1.
 * @param[in] s 并查集
 * @param[in] root1 一棵树的根
 * @param[in] root2 另一棵树的根
 * @return 如果二者已经在同一集合,并失败,返回-1,否则返回0
 */
int ufs_union(ufs_t *ufs, int root1, int root2) {
    if(root1 == root2) return -1;
    ufs->p[root1] += ufs->p[root2];
    ufs->p[root2] = root1;
    return 0;
}

/**
 * @brief 添加一对敌人.
 * @param[inout] s 并查集
 * @param[in] x 一对敌人的一个
 * @param[in] y 一对敌人的另一个
 * @return 无
 */
void ufs_add_opponent(ufs_t *ufs, int x, int y) {
    const int rx = ufs_find(ufs, x);
    const int ry = ufs_find(ufs, y);
    ufs_union(ufs, rx, ry);
    /* ry与y关系 + y与x的关系 + x与rx的关系 = ry与rx的关系 */
    ufs->dist[ry] = (ufs->dist[y] + 1 + ufs->dist[x]) % 2;
}

int main() {
    int T;

    scanf("%d", &T);
    while (T--) {
        ufs_t *ufs = ufs_create(MAXN);
        int n, m;
        char c;
        int x, y, rx, ry;
        scanf("%d%d%*c", &n, &m);

        while (m--) {
            scanf("%c%d%d%*c", &c, &x, &y); //注意输入
            rx = ufs_find(ufs, x);
            ry = ufs_find(ufs, y);

            if (c == 'A') {
                if (rx == ry) { //如果根节点相同,则表示能判断关系
                    if (ufs->dist[x] != ufs->dist[y])
                        printf("In different gangs.\n");
                    else
                        printf("In the same gang.\n");
                } else
                    printf("Not sure yet.\n");
            } else if (c == 'D') {
                ufs_add_opponent(ufs, x, y);
            }
        }
        ufs_destroy(ufs);
    }
    return 0;
}
\end{Codex}

\subsubsection{相关的题目}
与本题相同的题目:
\begindot
\item POJ 1703 Find them, Catch them, \myurl{http://poj.org/problem?id=1703}
\myenddot

与本题相似的题目:
\begindot
\item  None
\myenddot


\subsection{食物链} %%%%%%%%%%%%%%%%%%%%%%%%%%%%%%
\subsubsection{描述}
动物王国中有三类动物A,B,C,这三类动物的食物链构成了有趣的环形。A吃B, B吃C,C吃A。 现有$N$个动物,从1到$N$编号。每个动物都是A,B,C中的一种,但是我们并不知道它到底是哪一种。 

有人用两种说法对这N个动物所构成的食物链关系进行描述:
\begindot
\item 第一种说法是"1 X Y",表示X和Y是同类。 
\item 第二种说法是"2 X Y",表示X吃Y。 
\myenddot

此人对$N$个动物,用上述两种说法,一句接一句地说出$K$句话,这$K$句话有的是真的,有的是假的。当一句话满足下列三条之一时,这句话就是假话,否则就是真话。 
\begindot
\item 当前的话与前面的某些真的话冲突,就是假话; 
\item 当前的话中X或Y比N大,就是假话; 
\item 当前的话表示X吃X,就是假话。 
\myenddot

你的任务是根据给定的$N(1 \leq N \leq 50,000)$和$K$句话($0 \leq K \leq 100,000$),输出假话的总数。 

\subsubsection{输入}
第一行是两个整数$N$和$K$,以一个空格分隔。 

以下$K$行每行是三个正整数D,X,Y,两数之间用一个空格隔开,其中D表示说法的种类。
\begindot
\item 若D=1,则表示X和Y是同类。 
\item 若D=2,则表示X吃Y。
\myenddot

\subsubsection{输出}
只有一个整数,表示假话的数目。

\subsubsection{样例输入}
\begin{Code}
100 7
1 101 1 
2 1 2
2 2 3 
2 3 3 
1 1 3 
2 3 1 
1 5 5
\end{Code}

\subsubsection{样例输出}
\begin{Code}
3
\end{Code}

\subsubsection{分析}


\subsubsection{代码}
\begin{Codex}[label=food_chain.c]
/* POJ 1182 食物链, Catch them, http://poj.org/problem?id=1182 */
#include <stdio.h>
#include <stdlib.h>

/** 并查集. */
typedef struct ufs_t {
    int *p;     /** 树的双亲表示法 */
    int *dist;  /** 表示x与父节点p[x]的关系,0表示x与p[x]是同类,
                    1表示x吃p[x],2表示p[x]吃x */
    int size;   /** 大小. */
} ufs_t;

/**
 * @brief 创建并查集.
 * @param[in] ufs 并查集
 * @param[in] n 数组的容量
 * @return 并查集
 */
ufs_t* ufs_create(int n) {
    int i;
    ufs_t *ufs = (ufs_t*)malloc(sizeof(ufs_t));
    ufs->p = (int*)malloc(n * sizeof(int));
    ufs->dist = (int*)malloc(n * sizeof(int));
    for(i = 0; i < n; i++) {
        ufs->p[i] = -1;
        ufs->dist[i] = 0; // 自己与自己是同类
    }
    return ufs;
}

/**
 * @brief 销毁并查集.
 * @param[in] ufs 并查集
 * @return 无
 */
void ufs_destroy(ufs_t *ufs) {
    free(ufs->p);
    free(ufs->dist);
    free(ufs);
}

/**
 * @brief Find操作,带路径压缩,递归版.
 * @param[in] s 并查集
 * @param[in] x 要查找的元素
 * @return 包含元素x的树的根
 */
int ufs_find(ufs_t *ufs, int x) {
    if (ufs->p[x] < 0) return x; // 终止条件

    const int parent = ufs->p[x];
    ufs->p[x] = ufs_find(ufs, ufs->p[x]); /* 回溯时的压缩路径 */
    /* 更新关系 */
    ufs->dist[x] = (ufs->dist[x] + ufs->dist[parent]) % 3;
    return ufs->p[x];
}

/**
 * @brief Union操作,将root2并入到root1.
 * @param[in] s 并查集
 * @param[in] root1 一棵树的根
 * @param[in] root2 另一棵树的根
 * @return 如果二者已经在同一集合,并失败,返回-1,否则返回0
 */
int ufs_union(ufs_t *ufs, int root1, int root2) {
    if(root1 == root2) return -1;
    ufs->p[root1] += ufs->p[root2];
    ufs->p[root2] = root1;
    return 0;
}

/**
 * @brief 添加一对关系.
 * @param[inout] s 并查集
 * @param[in] x 一个
 * @param[in] y 另一个
 * @param[in] len
 * @return 无
 */
void ufs_add_relation(ufs_t *ufs, int x, int y, int relation) {
    const int rx = ufs_find(ufs, x);
    const int ry = ufs_find(ufs, y);
    ufs_union(ufs, ry, rx); /* 注意顺序! */
    /* rx与x关系 + x与y的关系 + y与ry的关系 = rx与ry的关系 */
    ufs->dist[rx] = (ufs->dist[y] - ufs->dist[x] + 3 + relation) % 3;
}

int main() {
    int n, k;
    int result = 0; /* 假话的数目 */
    ufs_t *ufs;

    scanf("%d%d", &n, &k);
    ufs = ufs_create(n + 1);

    while(k--) {
        int d, x, y;
        scanf("%d%d%d", &d, &x, &y);

        if (x > n || y > n || (d == 2 && x == y)) {
            result++;
        } else {
            const int rx = ufs_find(ufs, x);
            const int ry = ufs_find(ufs, y);

            if (rx == ry) { /* 若在同一个集合则可确定x和y的关系 */
                if((ufs->dist[x] - ufs->dist[y] + 3) % 3 != d - 1)
                    result++;
            } else {
                ufs_add_relation(ufs, x, y, d-1);
            }
        }
    }

    printf("%d\n", result);

    ufs_destroy(ufs);
    return 0;
}
\end{Codex}

\subsubsection{相关的题目}
与本题相同的题目:
\begindot
\item POJ 1182 食物链, \myurl{http://poj.org/problem?id=1182}
\item wikioi 1074 食物链, \myurl{http://www.wikioi.com/problem/1074/}
\myenddot

与本题相似的题目:
\begindot
\item  None
\myenddot


\section{线段树} %%%%%%%%%%%%%%%%%%%%%%%%%%%%%%

\subsection{原理和实现}
\textbf{线段树},也叫区间树(interval tree),它在各个节点保存一条线段(即子数组)。设数列$A$包含$N$个元素,则线段树的根节点表示整个区间$A[1,N]$,左孩子表示区间$A[1, (1+N)/2]$,右孩子表示区间$A[(1+N)/2+1, N]$,不断递归,直到叶子节点,叶子节点只包含一个元素。

线段树有如下特征:
\begindot
\item 线段树是一棵完全二叉树
\item 线段树的深度不超过$\log L$, $L$是区间的长度
\item 线段树把一个长度为L的区间分成不超过$2\log L$条线段
\myenddot

线段树的基本操作有构造线段树、区间查询和区间修改。

线段树通常用于解决和区间统计有关的问题。比如某些数据可以按区间进行划分,按区间动态进行修改,而且还需要按区间多次进行查询,那么使用线段树可以达到较快的查询速度。

用线段树解题,关键是要想清楚每个节点要存哪些信息(当然区间起点和终点,以及左右孩子指针是必须的),以及这些信息如何高效查询,更新。不要一更新就更新到叶子节点,那样更新操作的效率最坏有可能$O(N)$的。

\subsection{Balanced Lineup} %%%%%%%%%%%%%%%%%%%%%%%%%%%%%%
\subsubsection{描述}
给定$N(1 \leq N \leq 50,000)$ 个数, $A_1, A_2, ... , A_N$,求任意区间中最大数和最小数的差。

\subsubsection{输入}
第一行包含两个整数,$N$和$Q$。$Q$表示查询次数。

第2到N+1行,每行包含一个整数$A_i$。

第N+2到N+Q+1行,每行包含两个整数$a$和$b(1 \leq a \leq b \leq N)$,表示区间$A[a,b]$。

\subsubsection{输出}
对每个查询进行回应,输出该区间内最大数和最小数的差

\subsubsection{样例输入}
\begin{Code}
6 3
1
7
3
4
2
5
1 5
4 6
2 2
\end{Code}

\subsubsection{样例输出}
\begin{Code}
6
3
0
\end{Code}

\subsubsection{分析}
本题是“区间求和”,只需要“线段树构造”和“区间查询”两个操作。

\subsubsection{代码}
\begin{Codex}[label=balanced_lineup.c]
#include <stdio.h>
#include <stdlib.h>
#include <string.h>
#include <limits.h>

#define MAXN 50001
#define INF INT_MAX
#define max(a,b) ((a)>(b)?(a):(b))
#define min(a,b) ((a)<(b)?(a):(b))
#define L(a) ((a)<<1)
#define R(a) (((a)<<1)+1)

typedef struct node_t {
    int left, right;  /* 区间  */
    int max, min;  /* 本区间里的最大值和最小值 */
} node_t;

int A[MAXN]; /* 输入数据,0位置未用 */

/* 完全二叉树,结点编号从1开始,层次从1开始.
 * 用一维数组存储完全二叉树,空间约为4N,
 * 参考 http://comzyh.tk/blog/archives/479/
 */
node_t node[MAXN * 4];

int minx, maxx; /* 存放查询的结果 */

void init() {
    memset(node, 0, sizeof(node));
}

/* 以t为根结点,为区间A[l,r]建立线段树 */
void build(int t, int l, int r) {
    node[t].left = l, node[t].right = r;
    if (l == r) {
        node[t].max = node[t].min = A[l];
        return;
    }
    const int mid = (l + r) / 2;
    build(L(t), l, mid);
    build(R(t), mid + 1, r);
    node[t].max = max(node[L(t)].max,node[R(t)].max);
    node[t].min = min(node[L(t)].min,node[R(t)].min);
}

/* 查询根结点为t,区间为A[l,r]的最大值和最小值 */
void query(int t, int l, int r) {
    if (node[t].left == l && node[t].right == r) {
        if (maxx < node[t].max)
            maxx = node[t].max;
        if (minx > node[t].min)
            minx = node[t].min;
        return;
    }
    const int mid = (node[t].left + node[t].right) / 2;
    if (l > mid) {
        query(R(t), l, r);
    } else if (r <= mid) {
        query(L(t), l, r);
    } else {
        query(L(t), l, mid);
        query(R(t), mid + 1, r);
    }
}

int main() {
    int n, q, i;

    scanf("%d%d", &n, &q);
    for (i = 1; i <= n; i++) scanf("%d", &A[i]);

    init();
    /* 建立以tree[1]为根结点,区间为A[1,n]的线段树 */
    build(1, 1, n);

    while (q--) {
        int a, b;
        scanf("%d%d", &a, &b);
        maxx = 0;
        minx = INF;
        query(1, a, b); /* 查询区间A[a,b]的最大值和最小值 */
        printf("%d\n", maxx - minx);
    }
    return 0;
}
\end{Codex}

\subsubsection{相关的题目}
与本题相同的题目:
\begindot
\item POJ 3264 Balanced Lineup, \myurl{http://poj.org/problem?id=3264}
\myenddot

与本题相似的题目:
\begindot
\item  None
\myenddot


\subsection{线段树练习 1} %%%%%%%%%%%%%%%%%%%%%%%%%%%%%%
\subsubsection{描述}
一行$N(1\leq N < 100000)$个方格,开始每个格子里都有一个整数。现在动态地提出一些命令请求,有两种命令,查询和增加:求某一个特定的子区间$[a,b]$中所有元素的和;指定某一个格子$x$,加上一个特定的值A。现在要求你能对每个请求作出正确的回答。

\subsubsection{输入}
输入文件第一行为一个整数$N$,接下来是$n$行每行1个整数,表示格子中原来的整数。接下来是一个正整数$Q$,再接下来有$Q$行,表示$Q$个询问,第一个整数表示命令代号,命令代号1表示增加,后面的两个数$a$和$x$表示给位置$a$上的数值增加$x$,命令代号2表示区间求和,后面两个整数a和b,表示要求[a,b]之间的区间和。

\subsubsection{输出}
共$Q$行,每个整数

\subsubsection{样例输入}
\begin{Code}
6
4 
5 
6 
2 
1 
3
4
1 3 5
2 1 4
1 1 9
2 2 6
\end{Code}

\subsubsection{样例输出}
\begin{Code}
22
22
\end{Code}

\subsubsection{分析}
单点更新+区间求和

\subsubsection{代码}
\begin{Codex}[label=interval_tree1.c]
/* wikioi 1080 线段树练习 , http://www.wikioi.com/problem/1080/ */
#include <stdio.h>
#include <string.h>

#define L(a) ((a)<<1)
#define R(a) (((a)<<1)+1)
#define MAXN 100001

typedef long long int64_t;

typedef struct node_t {
    int left, right;
    int64_t sum;
} node_t;

int A[MAXN]; /* 输入数据,0位置未用 */

/* 完全二叉树,结点编号从1开始,层次从1开始.
 * 用一维数组存储完全二叉树,空间约为4N,
 * 参考 http://comzyh.tk/blog/archives/479/
 */
node_t node[MAXN * 4];

void init() {
    memset(node, 0, sizeof(node));
}

/* 以t为根结点,为区间A[l,r]建立线段树 */
void build(int t, int l, int r) {
    node[t].left = l;
    node[t].right = r;
    if (l == r) {
        node[t].sum = A[l];
        return;
    }
    const int mid = (l + r) / 2;
    build(L(t), l, mid);
    build(R(t), mid + 1, r);
    node[t].sum = node[L(t)].sum + node[R(t)].sum;
}

/* 给区间A[l,r]里的pos位置加delta */
void update(int t, int l, int r, int pos, int64_t delta) {
    if (node[t].left > pos || node[t].right < pos) return;
    if (node[t].left == node[t].right) {
        node[t].sum += delta;
        return;
    }

    const int mid = (node[t].left + node[t].right) / 2;
    if (l > mid) update(R(t), l, r, pos, delta);
    else if (r <= mid) update(L(t), l, r, pos, delta);
    else {
        update(L(t), l, mid, pos, delta);
        update(R(t), mid + 1, r, pos, delta);
    }
    node[t].sum = node[L(t)].sum + node[R(t)].sum;
}

/* 查询根结点为t,区间为A[l,r]的和 */
int64_t query(int t, int l, int r) {
    if (node[t].left == l && node[t].right == r)
        return node[t].sum;
    const int mid = (node[t].left + node[t].right) / 2;
    if (l > mid) return query(R(t), l, r);
    else if (r <= mid) return query(L(t), l, r);
    else return query(L(t), l, mid) + query(R(t), mid + 1, r);
}

int main() {
    int i, n, q;
    scanf("%d", &n);
    for (i = 1; i <= n; i++) scanf("%d", &A[i]);

    init();
    /* 建立以tree[1]为根结点,区间为A[1,n]的线段树 */
    build(1, 1, n);

    scanf("%d", &q);
    while (q--) {
        int cmd;
        scanf("%d", &cmd);
        if (cmd == 2) {
            int a, b;
            scanf("%d%d", &a, &b);
            printf("%lld\n", query(1, a, b)); /* 查询区间A[a,b]的和 */
        } else {
            int a;
            int64_t x;
            scanf("%d%lld", &a, &x);
            if (x != 0) update(1, 1, n, a, x);
        }
    }
    return 0;
}
\end{Codex}

\subsubsection{相关的题目}
与本题相同的题目:
\begindot
\item wikioi 1080 线段树练习 1, \myurl{http://www.wikioi.com/problem/1080/}
\myenddot

与本题相似的题目:
\begindot
\item  wikioi 1081 线段树练习 2, \myurl{http://www.wikioi.com/problem/1081/} 。本题是“区间更新+单点查询”,可以转化为线段树练习1。设原数组为$A[N]$,将其转化为差分数列,然后在数组上维护一棵线段树。“区间更新”操作转化为两个“单点更新”操作:将$A[a]$加上$x$,并将$A[b+1]$减去$x$(也就是加上$-x$)。“单点查询”操作转化为“区间求和”操作:求$A$数组$[1..i]$范围内所有数的和。这样就转化成与线段树练习1完全相同了。标程 \myurl{https://gist.github.com/soulmachine/6449609}
\myenddot


\subsection{A Simple Problem with Integers} %%%%%%%%%%%%%%%%%%%%%%%%%%%%%%
\subsubsection{描述}
You have $N$ integers, $A_1, A_2, ... , A_N$. You need to deal with two kinds of operations. One type of operation is to add some given number to each number in a given interval. The other is to ask for the sum of numbers in a given interval.

\subsubsection{输入}
The first line contains two numbers $N$ and $Q$. $1 \leq N,Q \leq 100000$.

The second line contains $N$ numbers, the initial values of $A_1, A_2, ... , A_N$. $-1000000000 \leq A_i \leq 1000000000$.

Each of the next $Q$ lines represents an operation.
"C a b c" means adding $c$ to each of $A_a, A_{a+1}, ... , A_b$. $-10000 ≤ c ≤ 10000$.
"Q a b" means querying the sum of $A_a, A_{a+1}, ... , A_b$.

\subsubsection{输出}
You need to answer all $Q$ commands in order. One answer in a line.

\subsubsection{样例输入}
\begin{Code}
10 5
1 2 3 4 5 6 7 8 9 10
Q 4 4
Q 1 10
Q 2 4
C 3 6 3
Q 2 4
\end{Code}

\subsubsection{样例输出}
\begin{Code}
4
55
9
15
\end{Code}

\subsubsection{提示}
The sums may exceed the range of 32-bit integers.

\subsubsection{分析}
区间更新+区间求和。

树节点要存哪些信息?只存该区间的和,行不行?只存和,会导致每次加数的时候都要更新到叶子节点,速度太慢。本题节点的结构如下:
\begin{Code}
typedef struct node_t {
    int left, right;
    int64_t sum;  /* 本区间的和实际上是sum+inc*[right-left+1] */
    int64_t inc;  /* 增量c的累加 */
} node_t;
\end{Code}

\subsubsection{代码}
\begin{Codex}[label=poj3468.c]
#include <stdio.h>
#include <string.h>

#define L(a) ((a)<<1)
#define R(a) (((a)<<1)+1)
#define MAXN 100001

typedef long long int64_t;

typedef struct node_t {
    int left, right;
    int64_t sum;  /* 本区间的和实际上是sum+inc*[right-left+1] */
    int64_t inc;  /* 增量c的累加 */
} node_t;

int A[MAXN]; /* 输入数据,0位置未用 */

/* 完全二叉树,结点编号从1开始,层次从1开始.
 * 用一维数组存储完全二叉树,空间约为4N,
 * 参考 http://comzyh.tk/blog/archives/479/
 */
node_t node[MAXN * 4];

void init() {
    memset(node, 0, sizeof(node));
}

/* 以t为根结点,为区间A[l,r]建立线段树 */
void build(int t, int l, int r) {
    node[t].left = l;
    node[t].right = r;
    if (l == r) {
        node[t].sum = A[l];
        return;
    }
    const int mid = (l + r) / 2;
    build(L(t), l, mid);
    build(R(t), mid+1, r);
    node[t].sum = node[L(t)].sum + node[R(t)].sum;
}

/* 给区间A[l,r]里的每个元素都加c */
void update(int t, int l, int r, int64_t c) {
    if (node[t].left == l && node[t].right == r) {
        node[t].inc += c;
        node[t].sum += c * (r - l + 1);
        return;
    }
    if (node[t].inc) {
        node[R(t)].inc += node[t].inc;
        node[L(t)].inc += node[t].inc;
        node[R(t)].sum += node[t].inc * (node[R(t)].right - node[R(t)].left + 1);
        node[L(t)].sum += node[t].inc * (node[L(t)].right - node[L(t)].left + 1);
        node[t].inc = 0;
    }
    const int mid = (node[t].left + node[t].right) / 2;
    if (l > mid)
        update(R(t), l, r, c);
    else if (r <= mid)
        update(L(t), l, r, c);
    else {
        update(L(t), l, mid, c);
        update(R(t), mid + 1, r, c);
    }
    node[t].sum = node[L(t)].sum + node[R(t)].sum;
}

/* 查询根结点为t,区间为A[l,r]的和 */
int64_t query(int t, int l, int r) {
    if (node[t].left == l && node[t].right == r)
        return node[t].sum;
    if (node[t].inc) {
        node[R(t)].inc += node[t].inc;
        node[L(t)].inc += node[t].inc;
        node[R(t)].sum += node[t].inc * (node[R(t)].right - node[R(t)].left + 1);
        node[L(t)].sum += node[t].inc * (node[L(t)].right - node[L(t)].left + 1);
        node[t].inc = 0;
    }
    const int mid = (node[t].left + node[t].right) / 2;
    if (l > mid)
        return query(R(t), l, r);
    else if (r <= mid)
        return query(L(t), l, r);
    else
        return query(L(t), l, mid) + query(R(t), mid + 1, r);
}

int main() {
    int i, n, q;
    char s[5];
    scanf("%d%d", &n, &q);
    for (i = 1; i <= n; i++) scanf("%d", &A[i]);

    init();
    /* 建立以tree[1]为根结点,区间为A[1,n]的线段树 */
    build(1, 1, n);

    while (q--) {
        int a, b;
        int64_t c;
        scanf("%s", s);
        if (s[0] == 'Q') {
            scanf("%d%d", &a, &b);
            printf("%lld\n", query(1, a, b)); /* 查询区间A[a,b]的和 */
        } else {
            scanf("%d%d%lld", &a, &b, &c);
            if (c != 0) update(1, a, b, c);
        }
    }
    return 0;
}
\end{Codex}

\subsubsection{相关的题目}
与本题相同的题目:
\begindot
\item POJ 3468 A Simple Problem with Integers, \myurl{http://poj.org/problem?id=3468}
\myenddot

与本题相似的题目:
\begindot
\item None
\myenddot


\subsection{约瑟夫问题} %%%%%%%%%%%%%%%%%%%%%%%%%%%%%%
\subsubsection{描述}
有编号从1到$N$的$N$个小朋友在玩一种出圈的游戏。开始时$N$个小朋友围成一圈,编号为$i+1$的小朋友站在编号为$i$小朋友左边。编号为1的小朋友站在编号为$N$的小朋友左边。首先编号为1的小朋友开始报数,接着站在左边的小朋友顺序报数,直到数到某个数字$M$时就出圈。直到只剩下1个小朋友,则游戏完毕。

现在给定$N,M$,求$N$个小朋友的出圈顺序。

\subsubsection{输入}
唯一的一行包含两个整数$N,M(1 \leq N,M \leq 30000)$。

\subsubsection{输出}
唯一的一行包含$N$个整数,每两个整数中间用空格隔开,第$i$个整数表示第$i$个出圈的小朋友的编号。

\subsubsection{样例输入}
\begin{Code}
5 3
\end{Code}

\subsubsection{样例输出}
\begin{Code}
3 1 5 2 4
\end{Code}

\subsubsection{分析}
约瑟夫问题的难点在于,每一轮都不能通过简单的运算得出下一轮谁淘汰,因为中间有人已经退出了。因此一般只能模拟,效率很低。

现在考虑,每一轮都令所有剩下的人从左到右重新编号,例如3退出后,场上还剩下1、2、4、5,则给1新编号1,2新编号2,4新编号3,5新编号4。不妨称这个编号为“剩余队列编号”。如下所示,括号内为原始编号:
\begin{Code}
1(1) 2(2) 3(3) 4(4) 5(5) --> 剩余队列编号3淘汰,对应原编号3
1(1) 2(2) 3(4) 4(5) --> 剩余队列编号1淘汰,对应原编号1
1(2) 2(4) 3(5) --> 剩余队列编号3淘汰,对应原编号5
1(2) 2(4) --> 剩余队列编号1淘汰,对应原编号2
1(4) --> 剩余队列编号1滔天,对应原编号4
\end{Code}

一个人在当前剩余队列中编号为$i$,则说明他是从左到右数第$i$个人,这启发我们可以用线段树来解决问题。用线段树维护原编号$[i..j]$内还有多少人没 有被淘汰,这样每次选出被淘汰者后,在当前线段树中查找位置就可以了。

例如我们有5个原编号,当前淘汰者在剩余队列中编号为3,先看左子树,即原编号[1..3]区间内,如果剩下的人不足3个,则说明当前剩余编号为3的 这个人原编号只能是在[4..5]区间内,继续在[4..5]上搜索;如果[1..3]内剩下的人大于等于3个,则说明就在[1..3]内,也继续缩小范围查找,这样即可在$O(\log N)$时间内完成对应。问题得到圆满的解决。

\subsubsection{代码}
\begin{Codex}[label=josephus_problem.c]
/* wikioi 1282 约瑟夫问题, http://www.wikioi.com/problem/1282/ */
#include <stdio.h>
#include <string.h>

#define L(a) ((a)<<1)
#define R(a) (((a)<<1)+1)
#define MAXN 30001

typedef struct node_t {
    int left, right;
    int count; /* 区间内的元素个数 */
} node_t;

/* 完全二叉树,结点编号从1开始,层次从1开始.
 * 用一维数组存储完全二叉树,空间约为4N,
 * 参考 http://comzyh.tk/blog/archives/479/
 */
node_t node[MAXN * 4];

void init() {
    memset(node, 0, sizeof(node));
}

/* 以t为根结点,为区间[l,r]建立线段树 */
void build(int t, int l, int r) {
    node[t].left = l;
    node[t].right = r;
    node[t].count = r - l + 1;
    if (l == r) return;

    const int mid = (r + l) / 2;
    build(L(t), l, mid);
    build(R(t), mid + 1, r);
}

/**
 * @brief 输出i
 * @param[in] t 根节点
 * @param[in] i 剩余队列编号
 * @return 被删除的实际数字
 */
int delete(int t, int i) {
    node[t].count--;
    if (node[t].left == node[t].right) {
        printf("%d ", node[t].left);
        return node[t].left;
    }
    if (node[L(t)].count >= i) return delete(L(t), i);
    else return delete(R(t), i - node[L(t)].count); /* 左子树人数不足,则在右子树查找 */
}

/**
 * @brief 返回 1到i内的活人数
 * @param[in] t 根节点
 * @param[in] i 原始队列的数字
 * @return 1到i内的活人数
 */
int get_count(int t, int i) {
    if (node[t].right <= i) return node[t].count;

    const int mid = (node[t].left + node[t].right) / 2;
    int s = 0;
    if (i > mid) {
        s += node[L(t)].count;
        s += get_count(R(t), i);
    } else
        s += get_count(L(t), i);
    return s;
}

int main() {
    int n, m;
    scanf("%d%d", &n, &m);

    init();
    build(1, 1, n);

    int i;
    int j = 0; /* 剩余队列的虚拟编号 */
    for (i = 1; i <= n; i++) {
        j += m;
        if (j > node[1].count)
            j %= node[1].count;
        if (j == 0) j = node[1].count;
        const int k = delete(1, j);
        j = get_count(1, k);
    }
    return 0;
}
\end{Codex}

\subsubsection{相关的题目}
与本题相同的题目:
\begindot
\item wikioi 1282 约瑟夫问题, \myurl{http://www.wikioi.com/problem/1282/}
\myenddot

与本题相似的题目:
\begindot
\item None
\myenddot


\section{Trie 树} %%%%%%%%%%%%%%%%%%%%%%%%%%%%%%


\subsection{原理和实现}

\begin{Codex}[label=trie_tree.c]
#include <stdio.h>
#include <string.h>
#include <stdlib.h>

#ifndef __cplusplus
typedef char bool;
#define false 0
#define true 1
#endif

#define MAXN 10000   /** 输入的编码的最大个数. */
#define CHAR_COUNT  10 /** 字符的种类,也即单个节点的子树的最大个数 */
#define MAX_CODE_LEN 10 /** 编码的最大长度. */
#define MAX_NODE_COUNT  (MAXN * MAX_CODE_LEN + 1)  /** 字典树的最大节点个数. */
                   /* 如果没有指定MAXN,则是 CHAR_COUNT^(MAX_CODE_LEN+1)-1 */

/** 字典树的节点 */
typedef struct trie_node_t {
    struct trie_node_t* next[CHAR_COUNT];
    bool is_tail; /** 标记当前字符是否位于某个串的尾部 */
} trie_node_t;

/** 字典树. */
typedef struct trie_tree_t {
    trie_node_t *root; /** 树的根节点 */
    int size; /** 树中实际出现的节点数 */

    trie_node_t nodes[MAX_NODE_COUNT]; /* 开一个大数组,加快速度 */
} trie_tree_t;

/** 创建. */
trie_tree_t* trie_tree_create(void) {
    trie_tree_t *tree = (trie_tree_t*)malloc(sizeof(trie_tree_t));
    tree->root = &(tree->nodes[0]);
    memset(tree->nodes, 0, sizeof(tree->nodes));
    tree->size = 1;
    return tree;
}

/** 销毁. */
void trie_tree_destroy(trie_tree_t *tree) {
    free(tree);
    tree = NULL;
}

/** 将当前字典树中的所有节点信息清空 */
void trie_tree_clear(trie_tree_t *tree) {
    memset(tree->nodes, 0, sizeof(tree->nodes));
    tree->size = 1; // 清空时一定要注意这一步!
}

/** 在当前树中插入word字符串,若出现非法,返回false */
bool trie_tree_insert(trie_tree_t *tree, char *word) {
    int i;
    trie_node_t *p = tree->root;
    while (*word) {
        int curword = *word - '0';
        if (p->next[curword] == NULL) {
            p->next[curword] = &(tree->nodes[tree->size++]);
        }
        p = p->next[curword];
        if (p->is_tail) return false; // 某串是当前串的前缀

        word++; // 指针下移
    }

    p->is_tail = true; // 标记当前串已是结尾

    // 判断当前串是否是某个串的前缀
    for (i = 0; i < CHAR_COUNT; i++)
        if (p->next[i] != NULL)
            return false;
    return true;
}
\end{Codex}


\subsection{Immediate Decodebility}


\subsubsection{描述}
An encoding of a set of symbols is said to be immediately decodable if no code for one symbol is the prefix of a code for another symbol. We will assume for this problem that all codes are in binary, that no two codes within a set of codes are the same, that each code has at least one bit and no more than ten bits, and that each set has at least two codes and no more than eight. 

Examples: Assume an alphabet that has symbols \fn{\{A, B, C, D\}}.

The following code is immediately decodable: 
\begin{Code}
A:01 B:10 C:0010 D:0000 
\end{Code}

but this one is not: 
\begin{Code}
A:01 B:10 C:010 D:0000 (Note that A is a prefix of C) 
\end{Code}


\subsubsection{输入}
Write a program that accepts as input a series of groups of records from standard input. Each record in a group contains a collection of zeroes and ones representing a binary code for a different symbol. Each group is followed by a single separator record containing a single 9; the separator records are not part of the group. Each group is independent of other groups; the codes in one group are not related to codes in any other group (that is, each group is to be processed independently).


\subsubsection{输出}
For each group, your program should determine whether the codes in that group are immediately decodable, and should print a single output line giving the group number and stating whether the group is, or is not, immediately decodable.

\subsubsection{样例输入}
\begin{Code}
01
10
0010
0000
9
01
10
010
0000
9
\end{Code}

\subsubsection{样例输出}
\begin{Code}
Set 1 is immediately decodable
Set 2 is not immediately decodable
\end{Code}

\subsubsection{分析}
判断一个串是否是另一个串的前缀,这正是Trie树(即字典树)的用武之地。


\subsubsection{代码}
\begin{Codex}[label=immediate_decodebility.c]
/* POJ 1056 IMMEDIATE DECODABILITY, http://poj.org/problem?id=1056 */

#define CHAR_COUNT  2
#define MAX_CODE_LEN 10
/** 字典树的最大节点个数.
 * 本题中每个code不超过10bit,即树的高度不超过11,因此最大节点个数为2^11-1
 */
#define MAX_NODE_COUNT  ((1<<(MAX_CODE_LEN+1))-1)

/* 等价于复制粘贴,这里为了节约篇幅,使用include,在OJ上提交时请用复制粘贴 */
#include "trie_tree.c"  /* 见“树->Trie树”这节 */

int main() {
    int T = 0;  // 测试用例编号
    char line[MAX_NODE_COUNT]; // 输入的一行
    trie_tree_t *trie_tree = trie_tree_create();
    bool islegal = true;

    while (scanf("%s", line) != EOF) {
        if (strcmp(line, "9") == 0) {
            if (islegal)
                printf("Set %d is immediately decodable\n", ++T);
            else
                printf("Set %d is not immediately decodable\n", ++T);
            trie_tree_clear(trie_tree);
            islegal = true;
        } else {
            if (islegal)
                islegal = trie_tree_insert(trie_tree, line);
        }
    }
    trie_tree_destroy(trie_tree);
    return 0;
}
\end{Codex}


\subsubsection{相关的题目}
与本题相同的题目:
\begindot
\item POJ 1056 IMMEDIATE DECODABILITY, \myurl{http://poj.org/problem?id=1056}
\myenddot

与本题相似的题目:
\begindot
\item POJ 3630 Phone List, \myurl{http://poj.org/problem?id=3630} \\参考代码 \myurl{https://gist.github.com/soulmachine/6609332}
\myenddot


\subsection{Hardwood Species}


\subsubsection{描述}
现在通过卫星扫描,扫描了很多区域的树,并获知了每棵树的种类,求每个种类的百分比。


\subsubsection{输入}
一行一棵树,表示该树的种类。每个名字不超过30字符,树的种类不超过10,000,不超过1,000,000棵树。


\subsubsection{输出}
按字母顺序,打印每个种类的百分比,精确到小数点后4位。

\subsubsection{样例输入}
\begin{Code}
Red Alder
Ash
Aspen
Basswood
Ash
Beech
Yellow Birch
Ash
Cherry
Cottonwood
Ash
Cypress
Red Elm
Gum
Hackberry
White Oak
Hickory
Pecan
Hard Maple
White Oak
Soft Maple
Red Oak
Red Oak
White Oak
Poplan
Sassafras
Sycamore
Black Walnut
Will
\end{Code}

\subsubsection{样例输出}
\begin{Code}
Ash 13.7931
Aspen 3.4483
Basswood 3.4483
Beech 3.4483
Black Walnut 3.4483
Cherry 3.4483
Cottonwood 3.4483
Cypress 3.4483
Gum 3.4483
Hackberry 3.4483
Hard Maple 3.4483
Hickory 3.4483
Pecan 3.4483
Poplan 3.4483
Red Alder 3.4483
Red Elm 3.4483
Red Oak 6.8966
Sassafras 3.4483
Soft Maple 3.4483
Sycamore 3.4483
White Oak 10.3448
Willow 3.4483
Yellow Birch 3.4483
\end{Code}

\subsubsection{分析}
无


\subsubsection{代码}
\begin{Codex}[label=hardwood_species.c]
/* POJ 2418 Hardwood Species, http://poj.org/problem?id=2418 */
#include <stdio.h>
#include <string.h>
#include <stdlib.h>

#ifndef __cplusplus
typedef char bool;
#define false 0
#define true 1
#endif

#define MAXN 1000   /**  no more than 10,000 species,会MLE,因此减一个0 */
#define CHAR_COUNT  128 /** ASCII 编码范围 */
#define MAX_WORD_LEN 30 /** 编码的最大长度. */
#define MAX_NODE_COUNT  (MAXN * MAX_WORD_LEN + 1)  /** 字典树的最大节点个数. */


/** 字典树的节点 */
typedef struct trie_node_t {
    struct trie_node_t* next[CHAR_COUNT];
    int count;  /** 该单词出现的次数 */
} trie_node_t;

/** 字典树. */
typedef struct trie_tree_t {
    trie_node_t *root; /** 树的根节点 */
    int size; /** 树中实际出现的节点数 */

    trie_node_t nodes[MAX_NODE_COUNT]; /* 开一个大数组,加快速度 */
} trie_tree_t;

/** 创建. */
trie_tree_t* trie_tree_create(void) {
    trie_tree_t *tree = (trie_tree_t*)malloc(sizeof(trie_tree_t));
    tree->root = &(tree->nodes[0]);
    memset(tree->nodes, 0, sizeof(tree->nodes));
    tree->size = 1;
    return tree;
}

/** 销毁. */
void trie_tree_destroy(trie_tree_t *tree) {
    free(tree);
    tree = NULL;
}

/** 将当前字典树中的所有节点信息清空 */
void trie_tree_clear(trie_tree_t *tree) {
    memset(tree->nodes, 0, sizeof(tree->nodes));
    tree->size = 1; // 清空时一定要注意这一步!
}

/** 在当前树中插入word字符串 */
void trie_tree_insert(trie_tree_t *tree, char *word) {
    trie_node_t *p = tree->root;
    while (*word) {
        if (p->next[*word] == NULL) {
            p->next[*word] = &(tree->nodes[tree->size++]);
        }
        p = p->next[*word];

        word++; // 指针下移
    }
    p->count++;
    return;
}


int n = 0;  // 输入的行数

/** 深度优先遍历. */
void dfs_travel(trie_node_t *root) {
    static char word[MAX_WORD_LEN + 1]; /* 中间结果 */
    static int pos;  /* 当前位置 */
    int i;

    if (root->count) { /* 如果count不为0,则肯定找到了一个单词 */
        word[pos] = '\0';
        printf("%s %0.4f\n", word, ((float)root->count * 100) / n);
    }
    for (i = 0; i < CHAR_COUNT; i++) {  /* 扩展 */
        if (root->next[i]) {
            word[pos++] = i;
            dfs_travel(root->next[i]);
            pos--; /* 返回上一层时恢复位置 */
        }
    }
}

int main() {
    char line[MAX_WORD_LEN + 1];
    trie_tree_t *trie_tree = trie_tree_create();

    while (gets(line)) {
        trie_tree_insert(trie_tree, line);
        n++;
    }
    dfs_travel(trie_tree->root);

    trie_tree_destroy(trie_tree);
    return 0;
}
\end{Codex}


\subsubsection{相关的题目}
与本题相同的题目:
\begindot
\item POJ 2418 Hardwood Species, \myurl{http://poj.org/problem?id=2418}
\myenddot

与本题相似的题目:
\begindot
\item 无
\myenddot
