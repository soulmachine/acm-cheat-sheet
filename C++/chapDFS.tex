\chapter{深度优先搜索}


\section{四色问题} %%%%%%%%%%%%%%%%%%%%%%%%%%%%%%

\subsubsection{描述}
给定$N(N \leq 8)$个点的地图,以及地图上各点的相邻关系,请输出用4种颜色将地图涂色的所有方案数(要求相邻两点不能涂成相同的颜色)。

数据中0代表不相邻,1代表相邻。

\subsubsection{输入}
第一行一个整数$N$,代表地图上有$N$个点。

接下来$N$行,每行$N$个整数,每个整数是0或者1。第i行第j列的值代表了第i个点和第j个点之间是相邻的还是不相邻,相邻就是1,不相邻就是0。我们保证a[i][j] = a[j][i]。

\subsubsection{输出}
染色的方案数

\subsubsection{样例输入}
\begin{Code}
8
0 0 0 1 0 0 1 0 
0 0 0 0 0 1 0 1 
0 0 0 0 0 0 1 0 
1 0 0 0 0 0 0 0 
0 0 0 0 0 0 0 0 
0 1 0 0 0 0 0 0 
1 0 1 0 0 0 0 0 
0 1 0 0 0 0 0 0
\end{Code}

\subsubsection{样例输出}
\begin{Code}
15552
\end{Code}

\subsubsection{分析}
这是一道经典的题目。深搜。

\subsubsection{代码}
\begin{Codex}[label=four_colors.c]
/* wikioi 1116 四色问题   , http://www.wikioi.com/problem/1116/ */
#include <stdio.h>
#include <string.h>

#define MAXN 8

int N;
int g[MAXN][MAXN];

/* 记录每个点的颜色,四种颜色用1234表示,0表示未染色. */
int history[MAXN];
int count; /* 方案个数 */

/**
 * 深搜,给第i个节点涂色.
 * @param i 第i个地点
 * @return 无
 */
void dfs(int i) {
    int j, c;
    if (i == N) {
        count++;
        return;
    }

    for (c = 1; c < 5; c++) {
        int ok = 1;
        for (j = 0; j < i; j++) {
            if (g[i][j] && c == history[j])
                ok = 0; /* 相邻且同色 */
        }
        if (ok) {
            history[i] = c;
            dfs(i + 1);
        }
    }
}

int main() {
    int i, j;

    scanf("%d", &N);
    for (i = 0; i < N; i++) {
        for (j = 0; j < N; j++) {
            scanf("%d", &g[i][j]);
        }
    }

    dfs(0);
    printf("%d\n", count);
    return 0;
}
\end{Codex}

\subsubsection{相关的题目}
与本题相同的题目:
\begindot
\item wikioi 1116 四色问题, \myurl{http://www.wikioi.com/problem/1116/}
\myenddot

与本题相似的题目:
\begindot
\item TODO
\myenddot


\section{全排列} %%%%%%%%%%%%%%%%%%%%%%%%%%%%%%

\subsubsection{描述}
给出一个正整数$n$, 请输出$n$的所有全排列

\subsubsection{输入}
一个整数$n(1 \leq n \leq 10)$

\subsubsection{输出}
一共$n!$行,每行$n$个用空格隔开的数,表示$n$的一个全排列。并且按全排列的字典序输出。

\subsubsection{样例输入}
\begin{Code}
3
\end{Code}

\subsubsection{样例输出}
\begin{Code}
1 2 3
1 3 2
2 1 3
2 3 1
3 1 2
3 2 1
\end{Code}

\subsubsection{分析}
这也是一道短小精悍的经典题目。深搜。从代码上看,与上一题的思路几乎一摸一样。

\subsubsection{代码}
\begin{Codex}[label=all_permutations.c]
/* wikioi 1294 全排列   , http://www.wikioi.com/problem/1294/ */
#include <stdio.h>
#include <string.h>

#define MAXN 10

int N;

int history[MAXN];
int count;

void dfs(int i) {
    int j, k;
    if (i == N) {
        count++;
        for (j = 0; j < N; j++) {
            printf("%d ", history[j]);
        }
        printf("\n");
        return;
    }

    for (k = 1; k <= N; k++) {
        int ok = 1;
        for (j = 0; j < i; j++) {
            if (history[j] == k)
                ok = 0;
        }
        if (ok) {
            history[i] = k;
            dfs(i + 1);
        }
    }
}

int main() {
    scanf("%d", &N);
    dfs(0);
    return 0;
}
\end{Codex}

\subsubsection{相关的题目}
与本题相同的题目:
\begindot
\item wikioi 1294 全排列 , \myurl{http://www.wikioi.com/problem/1294/}
\myenddot

与本题相似的题目:
\begindot
\item TODO
\myenddot
