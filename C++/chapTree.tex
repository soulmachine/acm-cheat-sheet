\chapter{树}

\section{二叉树的遍历} %%%%%%%%%%%%%%%%%%%%%%%%%%%%%%

\begin{Codex}[label=binary_tree.cpp]
#include <stack>
#include <queue>

 /*
  *@struct
  *@brief 二叉树结点
  */
typedef struct binary_tree_node_t {
    binary_tree_node_t *lchild;   /* 左孩子*/
    binary_tree_node_t *rchild;   /* 右孩子*/
    void* data; /* 结点的数据*/
}binary_tree_node_t;



/** 
  * @brief 先序遍历,递归.
  * @param[in] root 根结点
  * @param[in] visit 访问数据元素的函数指针
  * @return 无
  */
void pre_order_r(const binary_tree_node_t *root, 
                 int (*visit)(void*))
{
    if(root != NULL) {
        (void)visit(root->data);
        pre_order_r(root->lchild, visit);
        pre_order_r(root->rchild, visit);
    }
}

/** 
  * @brief 中序遍历,递归.
  */
void in_order_r(const binary_tree_node_t *root, 
                int (*visit)(void*))
{
    if(root != NULL) {
        pre_order_r(root->lchild, visit);
        (void)visit(root->data);
        pre_order_r(root->rchild, visit);
    }
}

/** 
  * @brief 后序遍历,递归.
  */
void post_order_r(const binary_tree_node_t *root, 
                  int (*visit)(void*))
{
    if(root != NULL) {
        pre_order_r(root->lchild, visit);
        pre_order_r(root->rchild, visit);
        (void)visit(root->data);
    }
}

/** 
 * @brief 先序遍历,非递归.
 */
void pre_order(const binary_tree_node_t *root, 
               int (*visit)(void*))
{
    const binary_tree_node_t *p;
    std::stack<const binary_tree_node_t *> s;

    p = root;

    if(p != NULL) {
        s.push(p);
    }

    while(!s.empty()) {
        p = s.top();
        s.pop();
        visit(p->data);
        if(p->rchild != NULL) {
            s.push(p->rchild);
        }
        if(p->lchild != NULL) {
            s.push(p->lchild);
        }
    }
}

/** 
 * @brief 中序遍历,非递归.
 */
void in_order(const binary_tree_node_t *root, 
              int (*visit)(void*))
{
    const binary_tree_node_t *p;
    std::stack<const binary_tree_node_t *> s;

    p = root;

    while(!s.empty() || p!=NULL) {
        if(p != NULL) {
            s.push(p);
            p = p->lchild;
        } else {
            p = s.top();
            s.pop();
            visit(p->data);
            p = p->rchild;
        }
    }
}

/** 
 * @brief 后序遍历,非递归.
 */
void post_order(const binary_tree_node_t *root, 
                int (*visit)(void*))
{
    /* p,正在访问的结点,q,刚刚访问过的结点*/
    const binary_tree_node_t *p, *q;
    std::stack<const binary_tree_node_t *> s;

    p = root;

    do {
        while(p != NULL) { /* 往左下走*/
            s.push(p);
            p = p->lchild;
        }
        q = NULL;
        while(!s.empty()) {
            p = s.top();
            s.pop();
            /* 右孩子不存在或已被访问,访问之*/
            if(p->rchild == q) {
                visit(p->data);
                q = p; /* 保存刚访问过的结点*/
            } else {
                /* 当前结点不能访问,需第二次进栈*/
                s.push(p);
                /* 先处理右子树*/
                p = p->rchild;
                break;
            }
        }
    }while(!s.empty());
}

/** 
 * @brief 层次遍历,也即BFS.
 *
 * 跟先序遍历一模一样,唯一的不同是栈换成了队列
 */
void level_order(const binary_tree_node_t *root, 
                 int (*visit)(void*))
{
    const binary_tree_node_t *p;
    std::queue<const binary_tree_node_t *> q;

    p = root;

    if(p != NULL) {
        q.push(p);
    }

    while(!q.empty()) {
        p = q.front();
        q.pop();
        visit(p->data);
        if(p->lchild != NULL) { /*先左后右或先右后左无所谓*/
            q.push(p->lchild);
        }
        if(p->rchild != NULL) {
            q.push(p->rchild);
        }
    }
}
\end{Codex}

\section{重建二叉树} %%%%%%%%%%%%%%%%%%%%%%%%%%%%%%
\begin{Codex}[label=binary_tree.cpp]
/** 
 * @brief 给定前序遍历和中序遍历,输出后序遍历.
 *
 * @param[in] n 序列的长度
 * @param[in] pre 前序遍历的序列
 * @param[in] in 中序遍历的序列
 * @param[out] post 后续遍历的序列
 * @return 无
 */
void build(const int n, const char * pre, const char *in, char *post) {
	if(n <= 0) return;
	int p = strchr(in, pre[0]) - in;
	build(p, pre + 1, in, post);
	build(n - p - 1, pre + p + 1, in + p + 1, post + p);
	post[n - 1] = pre[0];
}
\end{Codex}