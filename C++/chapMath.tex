\chapter{数学方法与常见模型}
数学对于算法很重要。没有好的数学基础,很难在算法上达到一定高度。本章介绍算法竞赛中常用的数学方法和模型。

\section{数论} %%%%%%%%%%%%%%%%%%%%%%%%%%%%%%

\subsection{欧几里德算法}

求最大公约数(greates common divisor)有很多方法,最经典的方法是欧几里德算法(Euclidean algorithm\footnote{\myurl{http://en.wikipedia.org/wiki/Euclid_algorithm}}),又称辗转相除法。

\begin{Codex}[label=gcd.c]
/**
 * @brief 求最大公约数,欧几里德算法,也即辗转相除法
 *
 * @param[in] a a, a > b > 0
 * @param[in] b b
 * @return a和b的最大公约数
 */
int gcd(int a, int b) {
    if (b == 0) return a;
    return gcd(b, a % b);
}

/**
 * @brief 求最大公约数,欧几里德算法,迭代版本
 *
 * @param[in] a a, a > b > 0
 * @param[in] b b
 * @return a和b的最大公约数
 */
int gcd1(int a, int b) {
    while (b != 0) {
        int tmp = b;
        b = a % b;
        a = tmp;
    }
    return a;
}

/**
 * @brief 求最大公约数,欧几里德算法,迭代版本,基于减法
 *
 * @param[in] a a, a > b > 0
 * @param[in] b b
 * @return a和b的最大公约数
 */
int gcd2(int a, int b) {
    while (a != b) {
        if (a > b) {
            a -= b;
        } else {
            b -= a;
        }
    }
    return a;
}
\end{Codex}

求出了最大公约数,可以利用它来求最小公倍数(least common multiple), $\text{lcm}(a,b)=a \times b / \text{gcd}(a,b)$。

\subsubsection{例题}
\begindot
\item wikioi 1212 最大公约数 ,\myurl{http://www.wikioi.com/problem/1212/}
\myenddot


\subsection{扩展欧几里德算法}
定理:对于不完全为$0$的非负整数$a,b$,必然存在整数对 $x,y$ ,使得 $gcd(a,b)=ax+by$。

这里$x$和$y$不一定是正数。扩展欧几里德算法(Extended Euclidean algorithm\footnote{\myurl{http://en.wikipedia.org/wiki/Extended_Euclidean_algorithm}})就是用来求$x$和$y$的。

\begin{Codex}[label=ex_gcd.c]
/**
 * @brief 扩展欧几里德算法
 * @param[in] a a, a > b > 0
 * @param[in] b b
 * @param[out] d d=gcd(a,b)
 * @param[out] x
 * @param[out] y
 * @return 无
 */
void ex_gcd(int a, int b, int *d, int *x, int *y) {
    if(b == 0) {
        *d = a; *x = 1; *y = 0;
    } else {
        ex_gcd(b, a % b, d, y, x); *y -= (*x)*(a/b);
    }
}
\end{Codex}

扩展欧几里德算法的应用主要有以下三个:
\begindot
\item 求解不定方程;
\item 求解模线性方程(线性同余方程);
\item 求解模的逆元;
\myenddot

\subsection{素数判定}
素数判定,又称素数测试(Primality test\footnote{\myurl{http://en.wikipedia.org/wiki/Primality_test}}),即给定一个正整数$n$,判断它是否是素数。

\subsubsection{暴力枚举法}
从2到$n$,依次作为除数,让$n$除以它们,只要有一个能整除$n$,则$n$不是素数。

\begin{Code}
/**
 * @brief 判断正整数n是否是素数
 * @param[in] n 正整数
 * @return 是,返回1,否,返回0
 */
int is_prime(int n) {
    int i;
    if (n < 2) return 0;
    for (i = 2; i < n; i++) {
        if (n % i == 0) return 0;
    }
    return 1;
}
\end{Code}

可以稍微做一点改进,从1到$\sqrt n$,依次作为除数。

\begin{Code}
/**
 * @brief 判断正整数n是否是素数,上界改为sqrt(n)
 * @param[in] n 正整数
 * @return 是,返回1,否,返回0
 */
int is_prime(int n) {
    int i;
    if (n < 2) return 0;
    const int upper = sqrt(n);

    for (i = 2; i <= upper; i++) {
        if (n % i == 0) return 0;
    }
    return 1;
}

/**
 * @brief 判断正整数n是否是素数,上界改为sqrt(n),但不使用sqrt()函数
 * @param[in] n 正整数
 * @return 是,返回1,否,返回0
 */
int is_prime1(int n) {
    int i;
    if (n < 2) return 0;
    for (i = 2; i*i <= n; i++) {
        if (n % i == 0) return 0;
    }
    return 1;
}
\end{Code}

\subsubsection{Eratosthenes筛法}
更高效的素数判定方法应该是预先计算出一张素数表,当判断一个数是否是素数时,直接查表即可。

怎样计算?用Eratosthenes筛法(Sieve of Eratosthenes\footnote{\myurl{http://en.wikipedia.org/wiki/Sieve_of_Eratosthenes}})

给出要筛数值的范围$n$,找出$\sqrt{n}$以内的素数$p_{1},p_{2},\dots,p_{k}$。先用2去筛,即把2留下,把2的倍数剔除掉;再用下一个质数,也就是3筛,把3留下,把3的倍数剔除掉;接下去用下一个质数5筛,把5留下,把5的倍数剔除掉;不断重复下去......。

\begin{Code}
/******** Eratosthenes筛法 ********/
#define MAXN 30000
/** prime_table[i]==1表示i是素数,等于0则不是素数 */
int prime_table[MAXN+1];

void compute_prime_table() {
    int i, j;
    const int upper = sqrt(MAXN);

    for (i = 2; i <= MAXN; i++) prime_table[i] = 1;
    prime_table[0] = 0;
    prime_table[1] = 0;

    for (i = 2; i < upper; i++) if(prime_table[i]) {
        for (j = 2; j * i <= MAXN; j++) prime_table[j*i] = 0;
    }
}

int is_prime(int n) {
    return prime_table[n];
}
\end{Code}

\subsubsection{暴力枚举法优化版}
来源未知。

\begin{Code}
int is_prime(unsigned long int n) {
    unsigned long int q, r, d;

    if (n < 32)
        return (0xa08a28acUL >> n) & 1;
    if ((n & 1) == 0)
        return 0;

    if (n % 3 == 0)
        return 0;
    if (n % 5 == 0)
        return 0;
    if (n % 7 == 0)
        return 0;

    for (d = 11;;) {
        q = n / d;
        r = n - q * d;
        if (q < d)
            return 1;
        if (r == 0)
            break;
        d += 2;
        q = n / d;
        r = n - q * d;
        if (q < d)
            return 1;
        if (r == 0)
            break;
        d += 4;
    }
    return 0;
}
\end{Code}


\subsubsection{例题}
\begindot
\item wikioi 1430 素数判定,\myurl{http://www.wikioi.com/problem/1430/}
\myenddot

\section{组合数学} %%%%%%%%%%%%%%%%%%%%%%%%%%%%%%

