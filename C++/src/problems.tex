% --------------------------------------------------------------
% This is all preamble stuff that you don't have to worry about.
% Head down to where it says "Start here"
% --------------------------------------------------------------
 
\documentclass[12pt]{article}
 
\usepackage[margin=1in]{geometry} 
\usepackage{amsmath,amsthm,amssymb}
\usepackage{mathtools}
\usepackage{upgreek}
\usepackage{algorithm}
\usepackage{comment}
\usepackage[noend]{algpseudocode}
 \usepackage{algpseudocode}
 \usepackage{pgffor}
 \DeclarePairedDelimiter\ceil{\lceil}{\rceil}
\DeclarePairedDelimiter\floor{\lfloor}{\rfloor}\newcommand{\N}{\mathbb{N}}
\newcommand{\Z}{\mathbb{Z}}
 
\newenvironment{theorem}[2][Theorem]{\begin{trivlist}
\item[\hskip \labelsep {\bfseries #1}\hskip \labelsep {\bfseries #2.}]}{\end{trivlist}}
\newenvironment{lemma}[2][Lemma]{\begin{trivlist}
\item[\hskip \labelsep {\bfseries #1}\hskip \labelsep {\bfseries #2.}]}{\end{trivlist}}
\newenvironment{exercise}[2][Exercise]{\begin{trivlist}
\item[\hskip \labelsep {\bfseries #1}\hskip \labelsep {\bfseries #2.}]}{\end{trivlist}}
\newenvironment{problem}[2][Problem]{\begin{trivlist}
\item[\hskip \labelsep {\bfseries #1}\hskip \labelsep {\bfseries #2.}]}{\end{trivlist}}
\newenvironment{question}[2][Question]{\begin{trivlist}
\item[\hskip \labelsep {\bfseries #1}\hskip \labelsep {\bfseries #2.}]}{\end{trivlist}}
\newenvironment{corollary}[2][Corollary]{\begin{trivlist}
\item[\hskip \labelsep {\bfseries #1}\hskip \labelsep {\bfseries #2.}]}{\end{trivlist}}
 
\begin{document}
 
% --------------------------------------------------------------
%                         Start here
% --------------------------------------------------------------
 
\title{Problems}%replace X with the appropriate number
\author{Jiankai Sun\\ %replace with your name
} %if necessary, replace with your course titles
 
\maketitle


\begin{question}{1}
Given $n$ intervals $(s_i,f_i)$ where $1 \leq i \leq n$, $s_i$ and $f_i$ is $i-th$ interval's start and finish time, its duration time is $f_i-s_i$. Use any algorithm to get the maximize the total sum of mutually disjoint intervals.
\end{question}
\begin{enumerate}
\item  we can use dynamic programming to solve this problem
\item for $n$ intervals $(s_i,f_i)$, it's sorted so that $f_1 \leq f_2 \leq ... \leq f_n$
\item We use $T(i)$ to represent the maximize total sum of intervals of $((s_1,f_1),(s_2,f_2),...(s_i,f_i))$. 
each interval has two choices: added to the select set or not.

 Define $p(i)$=the largest$ j<i$ that interval i doesn't overlap with $j$.

 So the recurrence function is:
$T(i)=max
\begin{cases}
T(p(i))+f_i-s_i \\
T(i-1); otherwise
\end{cases}
$
\item boundary condition $T(0)=0$
\item Implement the algorithm
\begin{algorithm}[H]
\caption{function maxSumInterval($i$) }

Sort the interval according to $f_i$, so that $f_1 \leq f_2 \leq ... \leq f_n$.

Start time array S=$(0,s_1,s_2,...,s_n)$, finish time array F=$(0,f_1,f_2,...,f_n)$.
\begin{algorithmic}[1]
\State global $T[0,1,2,...,n]$, T[0]=0; $P[1,...,n]$
\State 
\For	 {i $\leftarrow$ 1 to n}
\If{T(p(i))+$f_i$-$s_i$>T(i-1)}

T[i]=T(p(i))+$f_i$-$s_i$

P[i]=p(i)
\Else

T[i]=T[i-1]

P[i]=-1

\EndIf
\EndFor
\State Return T[n]
\end{algorithmic}
\end{algorithm}
\item we use $P[1,2,...,n]$ to print selected intervals

\begin{algorithm}[H]
\caption{function printInterval() }
\begin{algorithmic}[1]
\State global $P[1,...,n]$
\State i=n
\While{P[i] $\neq$ 0}
\If{P[i] $>$ 0}

print i


i=P[i]

\EndIf
\EndWhile

\State print i

\end{algorithmic}
\end{algorithm}
\item We use $O(nlog(n))$ to sort. For each $i$ we need $O(n)$ to scan all the interval list and repeat $n$ times, So the time complexity is $O(n^2)$
\item source code: max\_selected\_intervals.cpp
\end{enumerate}
\begin{question}{2}
Let $A=a_1a_2...a_m$ and $B=b_1b_2...b_n$ be two strings of characers. we
want to transform $A$ into $B$ using following operations:

delete a character

add a character

change a character

write a dynamic programming algorithm that finds the minimum number of operations needed to transform $A $into $B$
\end{question}
\begin{enumerate}
\item Let F[i,j] denotes the minimum number of operations needed to transform $a_1a_2...a_i$ to $b_1b_2...b_j$.
\item Recurrence relation

$
F[i,j]=or
\begin{cases}
F[i-1,j-1]; if a_i=b_j\\
min
\begin{cases}
F[i-1,j-1]+1; change\\
F[i,j-1]+1; add\\
F[i-1,j]+1; delete\\
\end{cases}
otherwise
\end{cases}
$

\item Boundary conditions

$F[0,k]=k$ for k in [0,n]

$F[k,0]=k$ for k in [0,m]
 
 \item implement the non recursive algorithm, see algorithm \ref{algminnum}

\begin{algorithm}[H]
\caption{function MinNumOper() }
\label{algminnum}
\begin{algorithmic}[1]
\State global $F[1...m;1..n]$, A[1...m], B[1...n]
\State Initialize $F[0,k]=k$ for k in [0,n] and $F[k,0]=k$ for k in [0,m]
 \For{i $\leftarrow$ 1 to m}
        \For {j $\leftarrow$ 1 to n}
        	\If {A[i]==B[j]} F[i,j]=F[i-1,j-1]
	\Else
	
	$
F[i,j]=min
\begin{cases}
F[i-1,j-1]+1; change\\
F[i,j-1]+1; add\\
F[i-1,j]+1; delete\\
\end{cases}
$
	\EndIf		
       \EndFor
\EndFor
 \State return F[m,n]
\end{algorithmic}
\end{algorithm} 
\item in the main function, we can call the function $MinNumOper()$ to to find the minimum number of operations needed to transform A into B
\item time complexity of this algorithm is $\Theta(mn)$
\item source code: min\_operation\_dp.cpp
\end{enumerate}
%--------------------------------------------------------------
%     You don't have to mess with anything below this line.
% --------------------------------------------------------------
 
\end{document}