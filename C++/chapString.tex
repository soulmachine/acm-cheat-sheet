\chapter{字符串}


\section{字符串排序} %%%%%%%%%%%%%%%%%%%%%%%%%%%%%%


\section{单词查找树} %%%%%%%%%%%%%%%%%%%%%%%%%%%%%%


\section{子串查找} %%%%%%%%%%%%%%%%%%%%%%%%%%%%%%
字符串的一种基本操作就是\textbf{子串查找}(substring search):给定一个长度为$N$的文本和一个长度为$M$的模式串(pattern string),在文本中找到一个与该模式相符的子字符串。

最简单的算法是暴力查找,时间复杂度是$O(MN)$。下面介绍两个更高效的算法。


\subsection{KMP算法}
KMP算法是Knuth、Morris和Pratt在1976年发表的。它的基本思想是,当出现不匹配时,就能知晓一部分文本的内容(因为在匹配失败之前它们已经和模式相匹配)。我们可以利用这些信息避免将指针回退到所有这些已知的字符之前。这样,当出现不匹配时,可以提前判断如何重新开始查找,而这种判断只取决于模式本身。

详细解释请参考《算法》\footnote{《算法》,Robert Sedgewick,人民邮电出版社,\myurl{http://book.douban.com/subject/10432347/}}第5.3.3节。这本书讲的是确定有限状态自动机(DFA)的方法。

推荐网上的几篇比较好的博客,讲的是部分匹配表(partial match table)的方法(即next数组),“字符串匹配的KMP算法” \myurl{http://t.cn/zTOPfdh},图文并茂,非常通俗易懂,作者是阮一峰;“KMP算法详解” \myurl{http://www.matrix67.com/blog/archives/115},作者是顾森 Matrix67;"Knuth-Morris-Pratt string matching" \myurl{http://www.ics.uci.edu/~eppstein/161/960227.html}。

使用next数组的KMP算法的C语言实现如下。
\begin{Codex}[label=kmp.c]
#include <stdio.h>
#include <stdlib.h>
#include <string.h>

/*
 * @brief 计算部分匹配表,即next数组.
 *
 * @param[in] pattern 模式串
 * @param[in] m 模式串的长度
 * @param[out] next next数组
 * @return 无
 */
void compute_prefix(const char pattern[], const int m, int next[]) {
    int i;
    int j = -1;

    next[0] = j;
    for (i = 1; i < m; i++) {
        while (j > -1 && pattern[j + 1] != pattern[i]) j = next[j];

        if (pattern[i] == pattern[j + 1]) j++;
        next[i] = j;
    }
}

/*
 * @brief KMP算法.
 *
 * @param[in] text 文本
 * @param[in] n 文本的长度
 * @param[in] pattern 模式串
 * @param[in] m 模式串的长度
 * @return 成功则返回第一次匹配的位置,失败则返回-1
 */
int kmp(const char text[], const int n, const char pattern[], const int m) {
    int i;
    int j = -1;
    int *next = (int*)malloc(sizeof(int) * m);

    compute_prefix(pattern, m, next);

    for (i = 0; i < n; i++) {
        while (j > -1 && pattern[j + 1] != text[i]) j = next[j];

        if (text[i] == pattern[j + 1]) j++;
        if (j == m - 1) {
            free(next);
            return i-j;
        }
    }

    free(next);
    return -1;
}


int main(int argc, char *argv[]) {
    char text[] = "ABC ABCDAB ABCDABCDABDE";
    char pattern[] = "ABCDABD";
    char *ch = text;
    int i = kmp(text, strlen(text), pattern, strlen(pattern));

    if (i >= 0) printf("matched @: %s\n", ch + i);
    return 0;
}
\end{Codex}


\subsection{Boyer-Moore算法}
详细解释请参考《算法》\footnote{《算法》,Robert Sedgewick,人民邮电出版社,\myurl{http://book.douban.com/subject/10432347/}}第5.3.4节。

推荐网上的几篇比较好的博客,“字符串匹配的Boyer-Moore算法” \myurl{http://www.ruanyifeng.com/blog/2013/05/boyer-moore_string_search_algorithm.html},图文并茂,非常通俗易懂,作者是阮一峰;Boyer-Moore algorithm, \myurl{http://www-igm.univ-mlv.fr/~lecroq/string/node14.html}。

有兴趣的读者还可以看原始论文\footnote{BOYER R.S., MOORE J.S., 1977, A fast string searching algorithm. Communications of the ACM. 20:762-772.}。

Boyer-Moore算法的C语言实现如下。
\begin{Codex}[label=boyer_moore.c]
/**
 * 本代码参考了 http://www-igm.univ-mlv.fr/~lecroq/string/node14.html
 * 精力有限的话,可以只计算坏字符的后移,好后缀的位移是可选的,因此可以删除
 * suffixes(), pre_gs() 函数
 */
#include <stdio.h>
#include <stdlib.h>
#include <string.h>

#define ASIZE 256  /* ASCII字母的种类 */

/*
 * @brief 预处理,计算每个字母最靠右的位置.
 *
 * @param[in] pattern 模式串
 * @param[in] m 模式串的长度
 * @param[out] right 每个字母最靠右的位置
 * @return 无
 */
static void pre_right(const char pattern[], const int m, int right[]) {
    int i;

    for (i = 0; i < ASIZE; ++i) right[i] = -1;
    for (i = 0; i < m; ++i) right[pattern[i]] = i;
}


static void suffixes(const char pattern[], const int m, int suff[]) {
    int f, g, i;

    suff[m - 1] = m;
    g = m - 1;
    for (i = m - 2; i >= 0; --i) {
        if (i > g && suff[i + m - 1 - f] < i - g)
            suff[i] = suff[i + m - 1 - f];
        else {
            if (i < g)
                g = i;
            f = i;
            while (g >= 0 && pattern[g] == pattern[g + m - 1 - f])
                --g;
            suff[i] = f - g;
        }
    }
}

/*
 * @brief 预处理,计算好后缀的后移位置.
 *
 * @param[in] pattern 模式串
 * @param[in] m 模式串的长度
 * @param[out] gs 好后缀的后移位置
 * @return 无
 */
static void pre_gs(const char pattern[], const int m, int gs[]) {
    int i, j;
    int *suff = (int*)malloc(sizeof(int) * (m + 1));

    suffixes(pattern, m, suff);

    for (i = 0; i < m; ++i) gs[i] = m;

    j = 0;
    for (i = m - 1; i >= 0; --i) if (suff[i] == i + 1)
        for (; j < m - 1 - i; ++j) if (gs[j] == m)
            gs[j] = m - 1 - i;
    for (i = 0; i <= m - 2; ++i) 
        gs[m - 1 - suff[i]] = m - 1 - i;
    free(suff);
}

/**
 * @brief Boyer-Moore算法.
 *
 * @param[in] text 文本
 * @param[in] n 文本的长度
 * @param[in] pattern 模式串
 * @param[in] m 模式串的长度
 * @return 成功则返回第一次匹配的位置,失败则返回-1
 */
int boyer_moore(const char text[], const int n, 
                const char pattern[], const int m) {
    int i, j;
    int right[ASIZE];  /* bad-character shift */
    int *gs = (int*)malloc(sizeof(int) * (m + 1));  /* good-suffix shift */

    /* Preprocessing */
    pre_right(pattern, m, right);
    pre_gs(pattern, m, gs);

    /* Searching */
    j = 0;
    while (j <= n - m) {
        for (i = m - 1; i >= 0 && pattern[i] == text[i + j]; --i);

        if (i < 0) { /* 找到一个匹配 */
            /* printf("%d ", j);
            j += bmGs[0]; */
            free(gs);
            return j;
        } else {
            const int max = gs[i] > right[text[i + j]] - m + 1 + i ?
                gs[i] : i - right[text[i + j]];
            j += max;
        }
    }
    free(gs);
    return -1;
}


int main() {
    const char text[]="HERE IS A SIMPLE EXAMPLE";
    const char pattern[] = "EXAMPLE";
    const int pos = boyer_moore(text, strlen(text), pattern, strlen(pattern));
    printf("%d\n", pos); /* 17 */
    return 0;
}
\end{Codex}


\subsection{Rabin-Karp算法}
详细解释请参考《算法》\footnote{《算法》,Robert Sedgewick,人民邮电出版社,\myurl{http://book.douban.com/subject/10432347/}}第5.3.5节。

Rabin-Karp算法的C语言实现如下。
\begin{Codex}[label=rabin_karp.c]
#include <stdio.h>
#include <string.h>

const int R = 256;  /** ASCII字母表的大小,R进制 */
/** 哈希表的大小,选用一个大素数,这里用16位整数范围内最大的素数 */
const long Q = 0xfff1;

/*
 * @brief 哈希函数.
 *
 * @param[in] key 待计算的字符串
 * @param[int] M 字符串的长度
 * @return 长度为M的子字符串的哈希值
 */
static long hash(const char key[], const int M) {
    int j;
    long h = 0;
    for (j = 0; j < M; ++j) h = (h * R + key[j]) % Q;
    return h;
}

/*
 * @brief 计算新的hash.
 *
 * @param[int] h 该段子字符串所对应的哈希值
 * @param[in] first 长度为M的子串的第一个字符
 * @param[in] next 长度为M的子串的下一个字符
 * @param[int] RM R^(M-1) % Q
 * @return 起始于位置i+1的M个字符的子字符串所对应的哈希值
 */
static long rehash(const long h, const char first, const char next, 
                   const long RM) {
    long newh = (h + Q - RM * first % Q) % Q;
    newh = (newh * R + next) % Q;
    return newh;
}

/*
 * @brief 用蒙特卡洛算法,判断两个字符串是否相等.
 * 
 * @param[in] pattern 模式串
 * @param[in] substring 原始文本长度为M的子串
 * @param[in] M 模式串的长度,也是substring的长度
 * @return 两个字符串相同,返回1,否则返回0
 */
static int check(const char pattern[], const char substring[], const int M) {
    return 1;
}

/**
 * @brief Rabin-Karp算法.
 *
 * @param[in] text 文本
 * @param[in] n 文本的长度
 * @param[in] pattern 模式串
 * @param[in] m 模式串的长度
 * @return 成功则返回第一次匹配的位置,失败则返回-1
 */
int rabin_karp(const char text[], const int n, 
               const char pattern[], const int m) {
    int i;
    const long pattern_hash = hash(pattern, m);
    long text_hash = hash(text, m);
    int RM = 1;
    for (i = 0; i < m - 1; ++i) RM = (RM * R) % Q;

    for (i = 0; i <= n - m; ++i) {
        if (text_hash == pattern_hash) {
            if (check(pattern, &text[i], m)) return i;
        }
        text_hash = rehash(text_hash, text[i], text[i + m], RM);
    }
    return -1;
}


int main() {
    const char text[]="HERE IS A SIMPLE EXAMPLE";
    const char pattern[] = "EXAMPLE";
    const int pos = rabin_karp(text, strlen(text), pattern, strlen(pattern));
    printf("%d\n", pos); /* 17 */
    return 0;
}
\end{Codex}

\subsection{总结}
\vspace{1ex}
\begin{center}
\begin{tabular}{llccccc}
\hline
\multirow{2}{*}{\textbf{算法}} & \multirow{2}{*}{\textbf{版本}} & \multicolumn{2}{c}{\textbf{复杂度}} & \textbf{在文本} & \multirow{2}{*}{\textbf{正确性}} & \textbf{辅助}\\
\cline{3-4} & & \textbf{最坏情况} & \textbf{平均情况} & \textbf{中回退} & & \textbf{空间}\\
\hline
\multirow{3}{*}{KMP算法} & 完整的DFA & 2N & 1.1N & 否 & 是 & MR\\
                         & 部分匹配表 & 3N & 1.1N & 否 & 是 & M\\
						 & 完整版本 & 3N & N/M & 是 & 是 & R\\
\hline
Boyer-Moore算法 & 坏字符向后位移 & MN & N/M & 是 & 是 & R\\
\hline
\multirow{2}{*}{Rabin-Karp算法$^*$} & 蒙特卡洛算法 & 7N & 7N & 否 & 是$^*$ & 1\\
                         & 拉斯维加斯算法 & $7N^*$ & 7N & 是 & 是 & 1\\
\hline
\end{tabular}
\end{center}
\small{* 概率保证,需要使用均匀和独立的散列函数}


\section{正则表达式} %%%%%%%%%%%%%%%%%%%%%%%%%%%%%%
